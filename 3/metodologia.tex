\chapter{METODOLOGÍA DE LA INVESTIGACIÓN}
\section{Diseño de la investigación}
En esta sección del documento se explicará cual es el diseño, el tipo y el enfoque del trabajo de
investigación, así como también la población y la muestra. 
%Para finalizar se explicará el
%proceso de aplicación de las redes neuronales convolucionales.
\subsection{Enfoque de la investigación}
El presente trabajo tendrá un enfoque cuantitativo ya que se busca diseñar y desarrollar instrumentos, en este caso modelos predictivos, para responder al problema estudiado a partir de medición de datos históricos en la plataforma Kickstarter con herramientas basadas en la estadística y matemáticas que puedan ser interpretadas por cualquier investigador. 

\subsection{Alcance de la investigación}
El alcance del presente trabajo será descriptivo ya que se recolectarán datos en un determinado rango de tiempo (desde 2009 hasta el presente año 2019) para describir el comportamiento de las campañas de proyectos tecnológicos en Kickstarter a partir de las características de sus variables y con ello, pronosticar su posible éxito o fracaso antes de finalizar la campaña con un nivel óptimo de precisión.

\subsection{Tipo de la investigación}
Para determinar el tipo de la investigación, primero es necesario definir el actual trabajo como Diseño Experimental ya que las variables que se tienen serán controladas, es decir, serán agregadas o quitadas en el o los modelos construidos en el experimento para analizar el impacto que este o estos tendrán en los resultados obtenidos. Dentro de esta categoría se clasifica como Diseño Experimental Puro ya que se busca medir la variable dependiente, en este caso Status (el estado actual del proyecto en Kickstarter) a partir de la manipulación de las demás variables independientes agregando o desagregándolas para comparar los rendimientos obtenidos de los instrumentos de medición y determinar cuáles de ellas finalmente serán tomadas en cuenta.

\subsection{Descripción del prototipo de investigación}
Teniendo como referencia y base principal el décimo antecedente explicado en el Capítulo II, la idea del prototipo final consistió en ensamblar las tres partes básicas de un proyecto: la primera consiste en el tratamiento de la metainformación (en la cual se realizarán, asimismo, tres experimentos independientes), el segundo, en el contenido visual y el último, el contenido textual respectivamente pero con el valor diferenciado de adaptar el modelo general de acuerdo a las variables y conjuntos de datos disponibles para el presente trabajo. Para ello, se representa cada una de las tres partes agrupadas en el marco de trabajo de la Figura \ref{3:fig1}.
\begin{figure}[htbp]
	\begin{center}
		\includegraphics[width=1\textwidth]{3/figures/prototipo.jpg}
		\caption{Marco de trabajo del prototipo final. Fuente: Elaboración propia}
		\label{3:fig1}
	\end{center}
\end{figure}

\section{Población y muestra}

\subsection{Población}
La población que será considerada para el presente trabajo será de 27,251 proyectos en Kickstarter de la categoría tecnología de todas las subcategorías entre los periodos 2009-2019, en su mayoría del territorio de los Estados Unidos de América.

\subsection{Muestra}
Debido a que se 214 imágenes del contenido visual no pudieron re-dimensionarse, así como 2 proyectos no contaban con descripciones en el contenido textual, se procedió a remover los 216 proyectos incompletos tanto en la metainformación como en las otras bases de datos, resultando finalmente en 27,035 registros en cada una de los tres conjuntos de datos. Sin embargo, la división en subconjuntos fue distinta en los tres casos y se dio de la siguiente manera:

\begin{itemize}
	\item Para la metainformación y el contenido textual, el conjunto de datos total de cada uno fue dividido en un subconjunto de entrenamiento (80\%) y uno de prueba (20\%) siguiendo las proporciones dadas en el octavo antecedente.
	\item Para el contenido visual, el conjunto de datos total fue dividido en tres subconjuntos: entrenamiento (80\%), validación (10\%) y prueba (10\%) siguiendo las proporciones dadas en el décimo antecedente.
\end{itemize}

\subsection{Unidad de análisis}
La unidad de análisis para el presente trabajo será un proyecto en Kickstarter de la categoría tecnología de cualquier subcategoría entre los periodos 2009-2019 dentro del territorio de los Estados Unidos de América.

\section{Operacionalización de Variables}
En la Tabla 5 se presentan las variables a usar para el conjunto de datos final basado en contenido textual y metainformación. Estas fueron seleccionadas de acuerdo al Benchmarking aplicado a los 10 antecedentes en el Capítulo II.

Los autores citados son los siguientes:
\begin{itemize}
	\item 
	\item 
	\item 
	\item 
	\item 
	\item 
	\item 
	\item 
\end{itemize}

\begin{table}[h!]
	\centering
	\small
	\begin{tabular}{ |m{3cm}|m{10cm}|m{2cm}|  }
		\hline
		\rowcolor{bluejean}
		\Centering \color{white}{Variable}& \Centering \color{white}{Detalle}& \Centering \color{white}{Tipo de dato}\\
		\hline
		\rowcolor{turq}
		\multicolumn{3}{c}{Variables independientes} \\
		\hline
		\textbf{backers\_count} & Número de patrocinadores de la campaña del proyecto. & int64 \\
		\hline
		\textbf{goal} &	Monto de la meta de financiamiento del proyecto. &	float64 \\
		\hline
		\textbf{duration} &	Duración de la campaña (en días). &	int64 \\
		\hline
		\textbf{description} &	Descripción del proyecto. &	object \\
		\hline
		\textbf{comments} & Comentarios de patrocinadores sobre el proyecto. & object \\
		\hline
		\rowcolor{turq}
		\multicolumn{3}{c}{Variable dependiente} \\
		\hline
		\textbf{state} & Estado de financiamiento del proyecto. & object \\
		\hline
	\end{tabular}
	\caption{Diccionario de datos del conjunto final a entrenar. Fuente: Elaboración propia.}
	\label{3:table1}
\end{table}

\section{Instrumentos de medida}


\section{Técnicas de recolección de datos}
Los conjuntos de datos recolectados para la investigación son un mix de observaciones cuantitativas (variables numéricas medibles como la meta de financiamiento, montos prometidos y duración de la campaña) y cualitativas (propaganda, descripción y comentarios del proyecto). La base de datos de la metainformación, que comprende las variables cuantitativas y propaganda del proyecto, fue consolidada luego de descargar un histórico público de 10 años de la página web “Web Robots”, fundada por los ex corporativos de TI Tomás Vitulskis y Paulius Jonaitis, y posteriormente pre-procesarla. Mientras que por el lado de la descripción y comentarios de cada proyecto, se usaron técnicas y herramientas de \textit{web scraping} a partir de los URLs. Para encontrar algunos de los papers con la información requerida más cercana, se utilizaron keywords o palabras clave como \textit{crowdfunding}, \textit{Machine Learning}, \textit{Deep Learning}, \textit{prediction}, \textit{Kickstarter}, \textit{accuracy} y \textit{projects}.


\section{Técnicas para el procesamiento y análisis de la información}



\section{Cronograma de actividades y presupuesto}
Se elaboró un cronograma de actividades de toda la investigación, mostrada en la Figura \ref{3:fig2}, contemplando desde el inicio de la misma, desarrollo, evaluación de resultados y sustentación en el mes de diciembre.
\begin{figure}[h]
	\begin{center}
		\includegraphics[width=1.1\textwidth]{3/figures/cronograma.jpg}
		\caption{Cronograma de actividades de la investigación. Fuente: Elaboración propia}
		\label{3:fig2}
	\end{center}
\end{figure}