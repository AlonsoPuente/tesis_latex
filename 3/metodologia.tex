\chapter{METODOLOGÍA DE LA INVESTIGACIÓN}
\section{Diseño de la investigación}
En esta sección del documento se explica cuál fue el diseño, el tipo y el enfoque del trabajo de investigación, así como también la población y la muestra. 

\subsection{Tipo de la investigación}
Para determinar el tipo de la investigación, primero fue necesario definir el actual trabajo como Diseño Experimental ya que, según \cite{bk_hernandez2014metodologia} en su libro \citetitle{bk_hernandez2014metodologia}, se pretende establecer el posible efecto de una causa que se manipula. Dentro de esta categoría se clasifica como Diseño Experimental Puro, ya que se manipuló intencionalmente más de una variable independiente (fueron agregadas y/o removidas) con la finalidad de medir el efecto que estas generan en la variable dependiente, el estado final de financiamiento de un proyecto de tecnología en Kickstarter.

\subsection{Enfoque de la investigación}
El presente trabajo tuvo un enfoque cuantitativo ya que, según \cite{bk_hernandez2014metodologia} en su libro \citetitle{bk_hernandez2014metodologia}, este enfoque utiliza la recolección de datos para probar hipótesis con base en la medición numérica y el análisis estadístico, con el fin de establecer pautas de comportamiento y probar teorías. Esto se refleja en los 10 pasos del proceso cuantitativo descritos por el anterior autor y que fueron aplicados en la investigación, desde la concepción de la idea hasta la elaboración del reporte de resultados.

\subsection{Población}
La población fueron todos los proyectos de la web de crowdfunding Kickstarter.

\subsection{Muestra}
La muestra fueron 27,251 proyectos, incluyendo exitosos y fracasados, de la categoría Tecnología, de la web de crowdfunding Kickstarter, entre los periodos 2009 y 2019. Como se mencionó en el Capítulo I, el criterio de la elección de esta categoría se debe a su bajo ratio de éxito de financiamiento en promedio durante los periodos mencionados, los cuales se consideraron hasta el 2019 dado que tomar el año 2020 incurriría en la distorsión del ratio por la coyuntura de la pandemia del Covid-19.

\subsection{Operacionalización de Variables}
En la Tabla \ref{3:table1} se presenta la operacionalización de variables para la investigación.

%\begin{table}[h!]

\begin{longtable}{M{5cm}M{4.5cm}M{5.5cm}}
	\caption[Matriz de operacionalización de variables]{Matriz de operacionalización de variables.}
	\label{3:table1}
	\newcommand{\multirot}[1]{\multirow{2}{*}[-8ex]{\rotcell{\rlap{#1}}}}
	\centering
	\small
	\tabularnewline \specialrule{.1em}{.05em}{.05em}
	VARIABLE & INDICADOR & CÁLCULO
	\\%%[5pt]
	\specialrule{.1em}{.05em}{.05em}
	\multirow{2}{5cm}[-2ex]{\centering Estado de financiamiento de un proyecto} & Proyecto exitoso & Monto meta $\leq$ monto contribuido                                                   \\%%[10pt]
	\cline{2-3} & Proyecto fracasado & Monto meta $>$ monto contribuido \\%%[10pt]
	\hline
	\multirow{2}{5cm}[-6ex]{\centering Aprendizaje Profundo Multimodal} & Modelo de Aprendizaje Profundo & \setlist{nolistsep}
	\begin{itemize}[noitemsep,leftmargin=*]
		\item Perceptrón Multicapa
		\item Red Neuronal Convolucional
		\item LSTM Bidireccional
	\end{itemize} \\
	\cline{2-3} 
	& Procesamiento de Lenguaje Natural &
	\setlist{nolistsep}
	\begin{itemize}[noitemsep,leftmargin=*]
		\item Pre-procesamiento de textos
		\item Representación de textos en vectores
	\end{itemize} \\
	\hline
	\multirow{5}{5cm}{\centering Efectividad del modelo} & Exactitud & $\frac{V.P.+V.N.}{V.P.+V.N.+F.P.+F.N.}$ \\%%[10pt]
	\cline{2-3} 
	& Precisión & $\frac{V.P.}{V.P.+F.P.}$ \\%%[10pt]
	\cline{2-3} 
	& Área bajo la curva ROC & $P( score(x^{+}) > score(x^{-}) )$ \\%%[10pt]
	\cline{2-3} 
	& Sensibilidad & $\frac{V.P.}{V.P.+F.N.}$ \\%%[10pt]
	\cline{2-3} 
	& Puntaje F1 & $\frac{2*Precisi\acute{o}n*Sensibilidad}{Precisi\acute{o}n+Sensibilidad}$ \\%%[10pt]
	\specialrule{.1em}{.05em}{.05em}
\end{longtable}
%\par	%%Salto de linea
%\bigskip
\begin{flushleft}	%%Alinear a la izquierda sin justificar
	\small Fuente: Elaboración propia.
\end{flushleft}
%\end{table}

El trabajo busca resolver el problema de clasificación del estado final de financiamiento de un proyecto implementando un modelo de Aprendizaje Profundo Multimodal.

Como se explica en la subsección 2.3.3, un proyecto de Kickstarter presenta 5 categorías de acuerdo al logro del objetivo de su meta: exitoso, fracasado, vigente, suspendido y cancelado. Dado que se busca predecir si, a partir de una serie de variables, el proyecto llegará a ser financiado o no, el estudio se delimita a estas 2 clases. Para determinar este valor, y según la política “Todo o Nada” de la plataforma explicado en la subsección 2.3.2, un proyecto es determinado como exitoso cuando el monto contribuido al culminar el tiempo de vida de su campaña logra ser mayor o igual a su meta estimada. De lo contrario, es un proyecto fracasado.

El modelo propuesto, alimentado de 3 modalidades independientes presentes en cada campaña de Kickstarter, comprende el ensamblaje de 3 modelos de Aprendizaje Profundo: un modelo Perceptrón Multicapa (MLP), una Red Neuronal Convolucional (CNN) y un modelo LSTM Bidireccional. Para estos 2 últimos, se trabajaron sus datos de entrada con técnicas de Procesamiento de Lenguaje Natural que incluye desde el pre-procesamiento de los textos hasta su conversión en vectores que finalmente fueron entrenados en sus respectivos modelos.

Por último, para poder evaluar la efectividad que este registrará al ser entrenado, su desempeño será evaluado a través de métricas de clasificación de Minería de Datos. Para la investigación, las 5 métricas utilizadas fueron la exactitud, la precisión, el área bajo la curva ROC, la sensibilidad y el puntaje F1, las cuales se describen en la Sección 2.2.9.

\section{Técnicas de recolección de datos}
Los conjuntos de datos recolectados para la investigación se componen de variables cuantitativas (características del proyecto como la meta de financiamiento, montos prometidos, duración de la campaña y otros indicadores financieros) y cualitativas (descripción y comentarios del proyecto). Para obtener los 3 principales datasets, se siguió el flujo de la Figura \ref{3:fig1}. La base de datos de la metainformación fue construída luego de descargar un histórico público de 10 años de la página web “Web Robots” y posteriormente pre-procesarla. Por el lado de la descripción y comentarios de los patrocinadores, se usaron técnicas y herramientas de \textit{web scraping} a partir de los URLs de cada proyecto disponibles en la Metainformación.

\begin{figure}[h]
	\begin{center}
		\includegraphics[width=0.89\textwidth]{3/figures/data_recolection_flux.png}
		\caption[Flujograma de la recolección de conjuntos finales de datos]{Flujograma de la recolección de conjuntos finales de datos.\\
			Fuente: Elaboración propia.}
		\label{3:fig1}
	\end{center}
\end{figure}

Asimismo, para encontrar algunos de los papers con la información requerida más cercana, se utilizaron keywords o palabras clave como \textit{crowdfunding}, \textit{Machine Learning}, \textit{Deep Learning}, \textit{prediction}, \textit{Kickstarter}, \textit{accuracy} y \textit{projects}.

%\newpage
\section{Técnicas para el procesamiento y análisis de la información}

\subsection{Metodología de implementación de la solución}
Como se explicó en el punto 2.2.6 del Marco Teórico del presente trabajo, según \cite{tec_braulio2015metodologiasdm}, dentro de los sistemas de analítica de negocio, Big Data y Minería de Datos, tres de las metodologías más usadas son CRISP-DM, SEMMA y KDD. Los mismos autores detallan en la Tabla \ref{2:table1} las características que cada una presenta al compararse.

Para escoger la metodología, se elaboró la Tabla \ref{3:table2} con el fin de comparar las utilizadas por los autores en los antecedentes. 17 de los 18 autores implementaron sus propias metodologías, las cuales de acuerdo a su similitud se agruparon en 8 grupos.

%\newpage
%\begin{table}[htbp]
\vspace{2ex}
\begingroup
	\renewcommand\arraystretch{0.2}
	\begin{longtable}{M{3cm}M{2.5cm}M{2.5cm}M{6cm}}
		\caption[Cuadro comparativo para la selección de la metodología]{Cuadro comparativo para la selección de la metodología.}
		\label{3:table2}
		\newcommand{\multirot}[1]{\multirow{2}{*}[-8ex]{\rotcell{\rlap{#1}}}}
		%\scriptsize
		\footnotesize
		\centering
		\small
		%% Se agrega tabularnewline para longtable
		\tabularnewline \specialrule{.1em}{.05em}{.05em}
		Metodología & Cantidad de referencias & Número de pasos & Nombre de los pasos
		\\
		\specialrule{.1em}{.05em}{.05em}
		{CRISP-DM}
		& 1
		& 6
		& \setlist{nolistsep}
		\begin{itemize}[label={--},nosep,noitemsep,leftmargin=*,topsep=0pt,partopsep=0pt]
			\item Comprensión del negocio.
			\item Comprensión de los datos.
			\item Preparación de los datos.
			\item Modelado.
			\item Evaluación.
			\item Despliegue.
		\end{itemize}                                                 
		\\
		\hline
		{Grupo A}
		& 2
		& 6
		& \setlist{nolistsep}
		\begin{itemize}[label={--},nosep,noitemsep,leftmargin=*,topsep=0pt,partopsep=0pt]
			\item Formulación del problema.
			\item Recolección de datos.
			\item Pre-procesamiento de datos.
			\item Modelado.
			\item Evaluación.
			\item Despliegue.
		\end{itemize} 
		\\
		\hline
		{Grupo B}
		& 1
		& 6
		& \setlist{nolistsep}
		\begin{itemize}[label={--},nosep,noitemsep,leftmargin=*,topsep=0pt,partopsep=0pt]
			\item Formulación del problema.
			\item Recolección de datos.
			\item Selección de características.
			\item Modelado.
			\item Evaluación.
			\item Despliegue.
		\end{itemize} 
		\\
		\hline
		{Grupo C}
		& 2
		& 5
		& \setlist{nolistsep}
		\begin{itemize}[label={--},nosep,noitemsep,leftmargin=*,topsep=0pt,partopsep=0pt]
			\item Recolección de datos.
			\item Pre-procesamiento de datos.
			\item Selección de características.
			\item Modelado.
			\item Evaluación.
		\end{itemize} 
		\\
		\hline
		{Grupo D}
		& 3
		& 5
		& \setlist{nolistsep}
		\begin{itemize}[label={--},nosep,noitemsep,leftmargin=*,topsep=0pt,partopsep=0pt]
			\item Formulación del problema.
			\item Recolección de datos.
			\item Pre-procesamiento de datos.
			\item Modelado.
			\item Evaluación.
		\end{itemize} 
		\\
		\hline
		{Grupo E}
		& 3
		& 5
		& \setlist{nolistsep}
		\begin{itemize}[label={--},nosep,noitemsep,leftmargin=*,topsep=0pt,partopsep=0pt]
			\item Recolección de datos.
			\item Pre-procesamiento de datos.
			\item Modelado.
			\item Evaluación.
			\item Despliegue.
		\end{itemize} 
		\\
		\hline
		{Grupo F}
		& 3
		& 4
		& \setlist{nolistsep}
		\begin{itemize}[label={--},nosep,noitemsep,leftmargin=*,topsep=0pt,partopsep=0pt]
			\item Recolección de datos.
			\item Pre-procesamiento de datos.
			\item Modelado.
			\item Evaluación.
		\end{itemize} 
		\\
		\hline
		{Grupo G}
		& 1
		& 4
		& \setlist{nolistsep}
		\begin{itemize}[label={--},nosep,noitemsep,leftmargin=*,topsep=0pt,partopsep=0pt]
			\item Recolección de datos.
			\item Modelado.
			\item Evaluación.
			\item Despliegue.
		\end{itemize} 
		\\
		\hline
		{Grupo H}
		& 2
		& 3
		& \setlist{nolistsep}
		\begin{itemize}[label={--},nosep,noitemsep,leftmargin=*,topsep=0pt,partopsep=0pt]
			\item Recolección de datos.
			\item Modelado.
			\item Evaluación.
		\end{itemize} 
		\\
		\specialrule{.1em}{.05em}{.05em}
	\end{longtable}%
\endgroup
	%\par	%%Salto de linea
	%\bigskip
	\begin{flushleft}	%%Alinear a la izquierda sin justificar
		\small Fuente: Elaboración propia
	\end{flushleft}
%\end{table}

Si bien es cierto que de las 3 metodologías mencionadas por \citeauthor{tec_braulio2015metodologiasdm}, solamente 1 autor especifica en su investigación el uso de una de ellas, en este caso CRISP-DM \parencite{pr_fernandezblanco2020crowdfunding_empirical}, otros 3 autores (grupos A y B) en sus propias metodologías siguieron secuencias de actividades que se encuentran comprendidas también dentro de esta metodología. Por su parte, los grupos C, F y H guardan cierta relación de semejanza con la metodología SEMMA por comprender enfocarse más en el modeloado, así como tener de primera actividad el Muestreo y culminar con la evaluación de resultados, obviando la actividad de formulación del problema. En el Anexo \ref{anexo4} se puede observar con más detalle las metodologías de la Tabla \ref{3:table2} asignadas a la investigación correspondiente.

Además de lo anterior expuesto, algunas de las siguientes características de acuerdo a la literatura también influyeron en la elección de la metodología CRISP-DM:

\begin{itemize}
	\item La metodología seleccionada contempla entre sus fases la comprensión del negocio además de la parte técnica que incluye el modelado y análisis de resultados. Esta guarda un rol importante ya que en ella se define el inicio de todo el proceso para dar el alcance del proyecto y definir objetivos que se buscan a partir de la Minería de Datos y Big Data. La formulación del problema se encuentra presente en el actual trabajo.
	\item Ayuda a los responsables del proyecto y/o investigación en la planeación y toma de decisiones (fase Despliegue) a partir de los resultados obtenidos, reportando y convirtiéndolos en oportunidades a considerar en los objetivos.
	\item Evalúa en todo el proceso los datos y variables usadas con el fin de crear el mejor modelo. Estas serán importantes para interpretar los resultados y tomar decisiones.
	\item Respecto a las otras dos metodologías, ambas omiten en su primera iteración la formulación del problema, la cual se encuentra presente en el actual trabajo. Por el lado de KDD, si bien esta contempla 9 pasos durante su proceso, el objetivo de KDD en cada uno de estos resulta ser más técnico, es decir, trabajar, seleccionar e interpretar métricas, variables, modelos, entre otros para obtener los mejores resultados más allá de considerar el contexto y comprensión del negocio. De hecho, no existe alguna fase dedicada al entendimiento del mismo.
	\item Por otra parte, la metodología SEMMA se basa, como su nombre lo indica, en la selección, exploración y modelado de grandes cantidades de datos para descubrir patrones de negocio desconocidos. Sin embargo, al limitarse a 5 fases comenzando con la fase de muestreo, no hace hincapié en la comprensión del negocio, sino más bien comienza con el procesamiento de datos para la construcción del modelo.
\end{itemize}

Cada una de las fases de la metodología seleccionada se detalla en la Figura \ref{3:fig2}.

\begin{figure}[htbp]
	\begin{center}
		\includegraphics[width=1\textwidth]{3/figures/metodologia.png}
		\caption[Metodología de la investigación]{Metodología de la investigación.\\
			Fuente: Elaboración propia}
		\label{3:fig2}
	\end{center}
\end{figure}

Durante el proceso de la implementación de la metodología, en donde se decidió construir un modelo de Aprendizaje Profundo Multimodal (utilizado por \cite{pr_kamath2018suplearn}, \cite{pr_jin2019dayssuccess}, y \cite{pr_cheng2019deeplearning}), de acuerdo a la literatura, se analizaron las siguientes posibles modalidades que se tomarían en cuenta para la investigación:

\begin{itemize}
	\item Metainformación: \cite{pr_chen2013kickpredict}, \cite{pr_mitra2014phrases}, \cite{pr_zhou2015projectdesc}, \cite{pr_chen2015predcrowd}, \cite{pr_beckwith2016predcrowd}, \cite{pr_li2016predcrowd}, \cite{pr_yuan2016textanalytics}, \cite{pr_sawhney2016usingLT}, \cite{pr_kaur2017socmedcrowd}, \cite{pr_kamath2018suplearn}, \cite{pr_yu2018deeplearning}, \cite{pr_jin2019dayssuccess}, \cite{pr_cheng2019deeplearning}, \cite{pr_fernandezblanco2020crowdfunding_empirical}.
	\item Descripción del proyecto: \cite{pr_mitra2014phrases}, \cite{pr_zhou2015projectdesc}, \cite{pr_yuan2016textanalytics}, \cite{pr_sawhney2016usingLT}, \cite{pr_kamath2018suplearn}, \cite{pr_lee2018contentDL}, \cite{pr_jin2019dayssuccess}, \cite{pr_cheng2019deeplearning}, \cite{pr_chen2019keywords_crowdfunding}, \cite{pr_chaichi2019nlp_3dprinting}.
	\item Comentarios en la campaña: \cite{pr_chen2015predcrowd}, \cite{pr_li2016predcrowd}, \cite{pr_lee2018contentDL}, \cite{pr_jin2019dayssuccess}, \cite{pr_shafqat2019topicpredictions}.
	\item Actualizaciones y/o recompensas: \cite{pr_zhou2015projectdesc}, \cite{pr_lee2018contentDL}.
	\item Contenido audiovisual (imagen y/o video del proyecto): \cite{pr_cheng2019deeplearning}.
\end{itemize}

De las anteriores modalidades, las actualizaciones no representan un cambio significativo ya que consisten en agregar o desagregar contenido textual o modificar recompensas en la campaña. Por el lado del contenido audiovisual, además de no presentarse en todas las campañas, para el caso particular de la categoría Tecnología no se encontró un patrón entre sus imágenes que pueda ser de utilidad, como se presenta en el trabajo de tesis de pregrado del presente autor \parencite{pr_puente2019kickstarter_prediction}.

Finalmente, fueron seleccionadas las modalidades de metainformación, descripción del proyecto y comentarios, para este último considerando solamente expresados por patrocinadores. Las 2 primeras se localizan en la sección principal de la campaña, mientras que la tercera se ubica en su sección respectiva, como se puede apreciar en la Figura \ref{3:fig3}.
\begin{figure}[htbp]
	\begin{center}
		\includegraphics[width=0.90\textwidth]{3/figures/framework.png}
		\caption[Marco de trabajo del prototipo final]{Marco de trabajo del prototipo final.\\
			Fuente: Elaboración propia}
		\label{3:fig3}
	\end{center}
\end{figure}

\newpage
\subsubsection{Comprensión del negocio}
En la primera fase, se definió el problema a partir de comprender el contexto de la investigación. A partir de aquí, se formularon los objetivos y requerimientos que se espera lograr en el proyecto. La lista de actividades y tareas se presenta en la Tabla \ref{3:table3}.

\vspace{2ex}
\begingroup
\renewcommand\arraystretch{0.3}
\begin{longtable}{m{5cm}m{5cm}M{5cm}}
	\caption[Actividades de fase Comprensión del negocio]{Actividades de fase Comprensión del negocio.}
	\label{3:table3}
	\newcommand{\multirot}[1]{\multirow{2}{*}[-8ex]{\rotcell{\rlap{#1}}}}
	%\scriptsize
	\footnotesize
	%\centering
	\small
	%% Se agrega tabularnewline para longtable
	\tabularnewline \specialrule{.1em}{.05em}{.05em}
	\centering Actividades & \centering Descripción & Tareas
	\\
	\specialrule{.1em}{.05em}{.05em}
	Definir problemas, objetivos e hipótesis.
	& Definición de los problemas, objetivos e hipótesis generales y específicos de la investigación.
	& \setlist{nolistsep}
	\begin{itemize}[label={--},nosep,noitemsep,leftmargin=*,topsep=0pt,partopsep=0pt]
		\item Identificar la realidad problemática del contexto.
		\item Definir los objetivos de la investigación.
		\item Formular las hipótesis del trabajo.
	\end{itemize}                                               
	\\
	\hline
	Desarrollar la literatura de la investigación.
	& Selección de los antecedentes y trabajos previos relacionados a la investigación.
	& \setlist{nolistsep}
	\begin{itemize}[label={--},nosep,noitemsep,leftmargin=*,topsep=0pt,partopsep=0pt]
		\item Buscar material académico sobre las variables de la investigación.
		\item Buscar material académico relacionado a la solución de la realidad problemática.
	\end{itemize} 
	\\
	\hline
	Definir metodología de la investigación.
	& Definición de la metodología a implementarse en el estudio.
	& \setlist{nolistsep}
	\begin{itemize}[label={--},nosep,noitemsep,leftmargin=*,topsep=0pt,partopsep=0pt]
		\item Analizar y comparar metodologías de trabajos previos.
		\item Seleccionar metodología a implementar en el trabajo.
	\end{itemize} 
	\\
	\specialrule{.1em}{.05em}{.05em}
\end{longtable}%
\endgroup

%\par	%%Salto de linea
%\bigskip
\begin{flushleft}	%%Alinear a la izquierda sin justificar
	\small Fuente: Elaboración propia
\end{flushleft}

A continuación, cada actividad de la anterior tabla es detallada y se indican sus entregables respectivos.

\newpage
\textbf{Actividad 1: Definir problemas, objetivos e hipótesis}
\\
En esta actividad se detalla cómo se definieron los problemas, objetivos e hipótesis generales y específicos de la investigación a partir del estudio de la coyuntura en que se desenvuelve la realidad problemática.

\textbf{Entregable}: Problemas, objetivos e hipótesis definidos.


\textbf{Actividad 2: Desarrollar la literatura de la investigación}
\\
A partir de la definición de los objetivos, se realizó la búsqueda de fuentes, publicaciones, libros, artículos y material académico relacionado al contexto del financiamiento colectivo, sus características y qué métodos se utilizaron para resolver problemas dentro del entorno. Asimismo, se elaboró el marco teórico abordado en el estudio.

\textbf{Entregable}: Material académico y conocimientos obtenidos sobre el ambiente del crowdfunding y la Inteligencia Artificial.


\textbf{Actividad 3: Definir metodología de la investigación}
\\
Luego de conseguir los recursos intelectuales suficientes y los trabajos previos para el desarrollo de la investigación, se analizó cada antecedente y se compararon todas las metodologías implementadas por cada autor. Al culminar con esta tarea, en base a la literatura y los procesos más alineados al presente trabajo, se seleccionó la metodología CRISP-DM como la mejor opción.

\textbf{Entregable}: Metodología de implementación de la solución.


\subsubsection{Comprensión de los datos}
La segunda fase abarcó tanto la recolección de los datos como el análisis estadístico de estos para conocer un poco más sus características. El conjunto total de datos utilizado para la investigación consistió en la recolección de 27,251 proyectos tecnológicos de Kickstarter comprendidos entre los años 2009 y 2019, en 3 partes: metainformación, descripción y comentarios (excluyendo los del creador) del proyectos.

Para ello, se elaboró una lista de actividades y tareas presentadas en la Tabla \ref{3:table4}.

\vspace{2ex}
\begingroup
\renewcommand\arraystretch{0.3}
\begin{longtable}{m{4cm}m{5.5cm}M{5.5cm}}
	\caption[Actividades de fase Comprensión de los datos]{Actividades de fase Comprensión de los datos.}
	\label{3:table4}
	\newcommand{\multirot}[1]{\multirow{2}{*}[-8ex]{\rotcell{\rlap{#1}}}}
	%\scriptsize
	\footnotesize
	%\centering
	\small
	%% Se agrega tabularnewline para longtable
	\tabularnewline \specialrule{.1em}{.05em}{.05em}
	\centering Actividades & \centering Descripción & Tareas
	\\
	\specialrule{.1em}{.05em}{.05em}
	Construir base de datos de Metainformación.
	& Construcción de base de datos que contenga las variables de la campaña con excepción del contenido textual (descripción, actualizaciones y comentarios acerca del proyecto).
	& \setlist{nolistsep}
	\begin{itemize}[label={--},nosep,noitemsep,leftmargin=*,topsep=0pt,partopsep=0pt]
		\item Descargar conjuntos de datos comprimidos de la página Web Robots.
		\item Descomprimir archivos en partes y fusionarlos en uno solo según su mes.
		\item Fusionar archivos por mes, realizar limpieza de datos y generar base final.
	\end{itemize} 
	\\
	\hline
	Construir base de datos de Descripción.
	& Construcción de base de datos que contenga la descripción del proyecto en la página principal de la campaña.
	& \setlist{nolistsep}
	\begin{itemize}[label={--},nosep,noitemsep,leftmargin=*,topsep=0pt,partopsep=0pt]
		\item Identificar sección de descripción a partir de la búsqueda del URL desde la base de datos de Metainformación.
		\item Capturar información de sección identificada.
		\item Generar base final.
	\end{itemize} 
	\\
	\hline
	Construir base de datos de Comentarios.
	& Construcción de base de datos que contenga los comentarios de los patrocinadores del proyecto en su sección correspondiente.
	& \setlist{nolistsep}
	\begin{itemize}[label={--},nosep,noitemsep,leftmargin=*,topsep=0pt,partopsep=0pt]
		\item Identificar sección de comentarios a partir de la búsqueda del URL desde la base de datos de Metainformación.
		\item Capturar información de sección identificada, excluyendo comentarios y respuestas de los creadores del proyecto.
		\item Generar base final.
	\end{itemize}
	\\
	\hline
	Realizar análisis exploratorio y estadístico de variables considerados.
	& Análisis exploratorio y estadístico de las variables disponibles para cada modalidad.
	& \setlist{nolistsep}
	\begin{itemize}[label={--},nosep,noitemsep,leftmargin=*,topsep=0pt,partopsep=0pt]
		\item Describir distribución de los datos y tendencias.
		\item Analizar existencia de correlación entre variables.
	\end{itemize}
	\\
	\specialrule{.1em}{.05em}{.05em}
\end{longtable}%
\endgroup

%\par	%%Salto de linea
%\bigskip
\begin{flushleft}	%%Alinear a la izquierda sin justificar
	\small Fuente: Elaboración propia
\end{flushleft}

A continuación, se detalla cada actividad y su entregable respectivo.

\textbf{Actividad 1: Construir base de datos de Metainformación}
\\
De acuerdo a la Figura \ref{3:fig1}, en esta actividad se enfocó en la construcción de la base de datos de la Metainformación desde la descarga de los conjuntos de datos comprimidos de la página Web Robots hasta la generación del archivo final luego de limpiar los datos.

La 2 primeras tareas de la anterior tabla se resumen en el flujo de la Figura \ref{3:fig4}.

\begin{figure}[h]
	\begin{center}
		\includegraphics[width=0.8\textwidth]{3/figures/flujograma_metadata_t1_t2.png}
		\caption[Proceso de obtención y preparación de datasets iniciales de Metainformación]{Proceso de obtención y preparación de datasets iniciales de Metainformación.\\
			Fuente: Elaboración propia.}
		\label{3:fig4}
	\end{center}
\end{figure}

La siguiente fase de la actividad fue la selección de variables y limpieza de datos utilizando el software Alteryx Designer, el cual permite desarrollar flujos de trabajo para preparar, unir y analizar volúmenes de datos complejos de distintas fuentes. Se ejecutó el flujograma representado de forma resumida en la Figura \ref{3:fig5}.

\begin{figure}[h]
	\begin{center}
		\includegraphics[width=1\textwidth,clip]{3/figures/flujograma_metadata_t3.png}
		\caption[Proceso resumido de generación de conjunto final de Metainformación]{Proceso resumido de generación de conjunto final de Metainformación.\\
			Fuente: Elaboración propia.}
		\label{3:fig5}
	\end{center}
\end{figure}

\textbf{Entregable}: Conjunto de datos de Metainformación, que contiene 19 variables cuantitativas y cualitativas para la muestra considerada.

\textbf{Actividad 2: Construir base de datos de Descripción}
\\
De acuerdo a la Figura \ref{3:fig1}, una vez generado el conjunto de datos de Metainformación descrito en la anterior actividad, a partir de la variable \textit{urls} se puede extraer la descripción de cada proyecto, siguiendo el proceso del diagrama de flujo representado en la Figura \ref{3:fig6}.

\begin{figure}[h]
	\begin{center}
		\includegraphics[width=0.5\textwidth]{3/figures/diagrama_flujo_scrapping_descripcion.png}
		\caption[Proceso de extracción de la descripción de un proyecto]{Proceso de extracción de la descripción de un proyecto.\\
			Fuente: Elaboración propia.}
		\label{3:fig6}
	\end{center}
\end{figure}

\textbf{Entregable}: Conjunto de datos de Descripción, que contiene la variable textual de descripción del proyecto para la muestra considerada.

\textbf{Actividad 3: Construir base de datos de Comentarios}
\\
De acuerdo a la Figura \ref{3:fig1}, y realizando el mismo ejercicio con Descripción, una vez generado el conjunto de datos de Metainformación, a partir de la variable \textit{urls} se pueden extraer los comentarios de los patrocinadores por cada proyecto, siguiendo el proceso del diagrama de flujo representado en la Figura \ref{3:fig7}.

\begin{figure}[h]
	\begin{center}
		\includegraphics[width=0.45\textwidth]{3/figures/diagrama_flujo_scrapping_comentarios.png}
		\caption[Proceso de extracción de comentarios de un proyecto]{Proceso de extracción de comentarios de un proyecto.\\
			Fuente: Elaboración propia.}
		\label{3:fig7}
	\end{center}
\end{figure}

\textbf{Entregable}: Conjunto de datos de Comentarios, que contiene la variable textual de comentarios de los patrocinadores, cada uno separado por autor y almacenados en conjunto en una lista por proyecto, para la muestra considerada.

\textbf{Actividad 4: Realizar análisis exploratorio y estadístico de variables considerados}
\\
En esta actividad, las variables de los 3 conjuntos finales de datos se describen en distribución de gráficos de barras, diagramas de cajas y bigotes y análisis de correlaciones de variables para la Metainformación; y nubes de palabras para la Descripción y Comentarios.

\textbf{Entregable}: Gráficos estadísticos de Metainformación, Descripción y Comentarios.

\subsubsection{Preparación de los datos}
En el anterior paso, se logró analizar estadísticamente las tendencias y distribuciones de cada variable según su respectiva modalidad. En esta etapa, como se detalla en la Tabla \ref{3:table5}, las variables observadas fueron pre-procesadas con la finalidad de incluir datos homologados que no afecten negativamente el rendimiento de los modelos.

\vspace{2ex}
\begingroup
\renewcommand\arraystretch{0.3}
\begin{longtable}{m{5cm}m{5cm}M{5cm}}
	\caption[Actividades de fase Preparación de los datos]{Actividades de fase Preparación de los datos.}
	\label{3:table5}
	\newcommand{\multirot}[1]{\multirow{2}{*}[-8ex]{\rotcell{\rlap{#1}}}}
	%\scriptsize
	\footnotesize
	%\centering
	\small
	%% Se agrega tabularnewline para longtable
	\tabularnewline \specialrule{.1em}{.05em}{.05em}
	\centering Actividades & \centering Descripción & Tareas
	\\
	\specialrule{.1em}{.05em}{.05em}
	Pre-procesar base de datos de Metainformación.
	& Pre-procesamiento de las variables actuales y adicionales luego de evaluación de su comportamiento entre ellas.
	& \setlist{nolistsep}
	\begin{itemize}[label={--},nosep,noitemsep,leftmargin=*,topsep=0pt,partopsep=0pt]
		\item Añadir más variables cuantitativas según literatura.
		\item Evaluar comportamiento de variables en conjunto utilizando matriz de correlaciones.
		\item Armar grupos combinatorias potenciales de variables.
		\item Escalar variables a un mismo rango.
	\end{itemize}                                             
	\\
	\hline
	Pre-procesar base de datos de Descripción.
	& Pre-procesamiento de la variable \textit{description}.
	& \setlist{nolistsep}
	\begin{itemize}[label={--},nosep,noitemsep,leftmargin=*,topsep=0pt,partopsep=0pt]
		\item Realizar limpieza de datos de los textos.
		\item Armar vocabulario de palabras únicas.
		\item Transformar textos en vectores de palabras.
	\end{itemize} 
	\\
	\hline
	Pre-procesar base de datos de Comentarios.
	& Pre-procesamiento de la variable \textit{comments}.
	& \setlist{nolistsep}
	\begin{itemize}[label={--},nosep,noitemsep,leftmargin=*,topsep=0pt,partopsep=0pt]
		\item Realizar limpieza de datos de los textos.
		\item Armar vocabulario de palabras únicas.
		\item Transformar textos en vectores de palabras.
	\end{itemize}
	\\
	\specialrule{.1em}{.05em}{.05em}
\end{longtable}%
\endgroup

%\par	%%Salto de linea
%\bigskip
\begin{flushleft}	%%Alinear a la izquierda sin justificar
	\small Fuente: Elaboración propia
\end{flushleft}

A continuación, se detalla cada actividad y su entregable respectivo.

\textbf{Actividad 1: Pre-procesar base de datos de Metainformación}
\\
En esta actividad, antes de realizar el pre-procesamiento de las variables finales, se agregaron algunas cuantitativas sugeridas en la literatura que no estaban consideradas inicialmente. Para evaluar correctamente el nuevo conjunto de variables, se realizó un nuevo análisis de correlación y se armaron grupos de combinatorias de variables cuyos rendimientos serán medidos en los siguientes pasos. Previamente a la evaluación, los subconjuntos de entrenamiento y prueba divididos en 80\% y 20\% respectivamente (según \cite{pr_yuan2016textanalytics}, \cite{pr_yu2018deeplearning}, \cite{pr_chen2019keywords_crowdfunding}, \cite{pr_mitra2014phrases} y \cite{pr_sawhney2016usingLT}) estratificados según la variable \textit{state}, se escalaron a un mismo rango para evitar un entrenamiento incorrecto.

\textbf{Entregable}: Conjuntos de datos pre-procesados de Metainformación.

\textbf{Actividad 2: Pre-procesar base de datos de Descripción}
\\
Esta actividad se resume en la Figura \ref{3:fig8}, en donde se realizó desde la limpieza de los datos textuales, eliminando palabras y partes de ellas que no aportan para el diccionario final que se usará en la siguiente etapa, hasta la elaboración del vocabulario de palabras únicas, mayormente sustantivos y verbos en su forma original, los cuales se les asignaron un código para poder formar vectores de palabras y crear más adelante, las incrustaciones de palabras.

\begin{figure}[!ht]
	\begin{center}
		\includegraphics[width=0.33\textwidth]{3/figures/description_data_clean.png}
		\caption[Proceso de limpieza de conjunto de datos de descripciones]{Proceso de limpieza de conjunto de datos de descripciones.\\
			Fuente: Elaboración propia.}
		\label{3:fig8}
	\end{center}
\end{figure}

Luego de separar, al igual que la Metainformación, los subconjuntos de entrenamiento y prueba en 80\% y 20\% respectivamente estratificados según la variable \textit{state}, se ejecutó el proceso de la Figura \ref{3:fig9} dividido en 2 partes.

\begin{figure}[!ht]
	\centering
	\small
	\begin{subfigure}{.50\textwidth}
		\centering
		\includegraphics[width=0.65\linewidth]{3/figures/description_text_to_sequence.png}
		\caption{vectorización de palabras}
	\end{subfigure}%
	\begin{subfigure}{.50\textwidth}
		\centering
		\includegraphics[width=0.55\linewidth]{3/figures/description_embedding_matrix.png}
		\caption{matriz de incrustaciones de palabras}
	\end{subfigure}
	\caption[Procesos de vectorización y creación de matriz incrustaciones de palabras]{Procesos de vectorización y creación de matriz incrustaciones de palabras.\\
		Fuente: Elaboración propia.}
	\label{3:fig9}
\end{figure}

La primera mitad corresponde a la transformación del texto, para los subconjuntos de entrenamiento y prueba, de las descripciones en representaciones de vectores numéricos para la fase de entrenamiento del modelo. De forma paralela, se creó una matriz de incrustaciones de palabras que será parte de la capa de incrustaciones en la arquitectura del modelo.

\textbf{Entregable}: Descripciones representadas en secuencias de vectores numéricos y matriz de incrustaciones de palabras.

\textbf{Actividad 3: Pre-procesar base de datos de Comentarios}
\\
Al igual que en el proceso de Descripción, se realizó la limpieza de los datos textuales siguiendo la misma secuencia de la Figura \ref{3:fig7} para generar la representación de palabras de los comentarios en vectores numéricos, con la diferencia de acotar el tamaño final de los vectores debido a su extensa longitud al agruparse todos aquellos separados por autor. De igual manera, para generar la matriz de incrustaciones de palabras, se repitió el proceso de la Figura \ref{3:fig9}.

\textbf{Entregable}: Comentarios representadas en secuencias de vectores numéricos y matriz de incrustaciones de palabras.

\subsubsection{Modelamiento}
Al concluir la etapa del pre-procesamiento, las variables pudieron ser utilizadas para los modelos de cada modalidad, que fueron diseñados siguiendo las actividades de la Tabla \ref{3:table6}.

\vspace{2ex}
\begingroup
\renewcommand\arraystretch{0.3}
\begin{longtable}{m{4.6cm}m{4.9cm}M{5.5cm}}
	\caption[Actividades de fase Modelamiento]{Actividades de fase Modelamiento.}
	\label{3:table6}
	\newcommand{\multirot}[1]{\multirow{2}{*}[-8ex]{\rotcell{\rlap{#1}}}}
	%\scriptsize
	\footnotesize
	% \centering
	\small
	%% Se agrega tabularnewline para longtable
	\tabularnewline \specialrule{.1em}{.05em}{.05em}
	\centering Actividades & \centering Descripción & Tareas
	\\
	\specialrule{.1em}{.05em}{.05em}
	Desarrollar modelo predictivo de Metainformación.
	& Implementación del modelo Perceptrón Multicapa (MLP) para la modalidad Metainformación.
	& \setlist{nolistsep}
	\begin{itemize}[label={--},nosep,noitemsep,leftmargin=*,topsep=0pt,partopsep=0pt]
		\item Diseñar arquitectura del modelo predictivo.
		\item Entrenar cada grupo de combinatoria de variables con arquitectura diseñada.
		\item Realizar experimentos con hiperparámetros modificados.
		\item Comparar resultados por cada grupo.
	\end{itemize}
	\\
	\hline
	Desarrollar modelo predictivo de Descripción.
	& Implementación de la Red Neuronal Convolucional (CNN) para la modalidad Descripción.
	& \setlist{nolistsep}
	\begin{itemize}[label={--},nosep,noitemsep,leftmargin=*,topsep=0pt,partopsep=0pt]
		\item Diseñar arquitectura del modelo predictivo.
		\item Entrenar base de datos pre-procesada con arquitectura diseñada.
		\item Realizar experimentos con hiperparámetros modificados.
		\item Comparar resultados.
	\end{itemize}
	\\
	\hline
	Desarrollar modelo predictivo de Comentarios.
	& Implementación del modelo LSTM Bidireccional para la modalidad Comentarios.
	& \setlist{nolistsep}
	\begin{itemize}[label={--},nosep,noitemsep,leftmargin=*,topsep=0pt,partopsep=0pt]
		\item Diseñar arquitectura del modelo predictivo.
		\item Entrenar base de datos pre-procesada con arquitectura diseñada.
		\item Realizar experimentos con hiperparámetros modificados.
		\item Comparar resultados.
	\end{itemize}
	\\
	\hline
	Desarrollar modelo predictivo ensamblado.
	& Implementación del modelo de Aprendizaje Profundo Multimodal.
	& \setlist{nolistsep}
	\begin{itemize}[label={--},nosep,noitemsep,leftmargin=*,topsep=0pt,partopsep=0pt]
		\item Diseñar arquitectura del modelo predictivo ensamblado.
		\item Entrenar modelo con los mejores resultados por cada modalidad.
		\item Realizar experimentos con hiperparámetros modificados.
		\item Comparar resultados.
	\end{itemize}
	\\
	\specialrule{.1em}{.05em}{.05em}
\end{longtable}%
\endgroup

%\par	%%Salto de linea
%\bigskip
\begin{flushleft}	%%Alinear a la izquierda sin justificar
	\small Fuente: Elaboración propia
\end{flushleft}

A continuación, se detalla cada actividad y su entregable respectivo.

\textbf{Actividad 1: Desarrollar modelo predictivo de Metainformación}
\\
En esta actividad, se diseñó la arquitectura que se usó para la modalidad Metainformación. Para ello, se realizó benchmarking sobre las propuestas de los autores enunciadas a continuación.

\begin{itemize}
	\item Perceptrón Multicapa (MLP): \cite{pr_kamath2018suplearn}, \cite{pr_yu2018deeplearning}, \cite{pr_cheng2019deeplearning}.
	\item Máquina de Vectores de Soporte (SVM): \cite{pr_chen2013kickpredict}, \cite{pr_beckwith2016predcrowd}, \cite{pr_sawhney2016usingLT}.
	\item Regresión Logística: \cite{pr_mitra2014phrases}, \cite{pr_zhou2015projectdesc}, \cite{pr_beckwith2016predcrowd}, \cite{pr_li2016predcrowd}, \cite{pr_kaur2017socmedcrowd}.
	\item Regresión Log-logística: \cite{pr_li2016predcrowd}.
	\item Bosques Aleatorios: \cite{pr_chen2015predcrowd}, \cite{pr_yuan2016textanalytics}, \cite{pr_kamath2018suplearn}.
	\item Árboles de Decisión: \cite{pr_beckwith2016predcrowd}, \cite{pr_kamath2018suplearn}.
	\item Naïve Bayes: \cite{pr_beckwith2016predcrowd}, \cite{pr_kamath2018suplearn}.
	\item Modelo Seq2seq: \cite{pr_jin2019dayssuccess}.
\end{itemize}

De acuerdo a los resultados de los autores citados, los modelos de Redes Neuronales (MLP), SVM y Regresión Logística tuvieron los mejores rendimientos en sus experimentos. Dado que la investigación consistió en implementar un modelo de Aprendizaje Profundo Multimodal, en donde se ensamblan modelos de Aprendizaje Profundo, se optó por el modelo Perceptrón Multicapa para la Metainformación.

\textbf{Entregable}: Modelo Perceptrón Multicapa para modalidad Metainformación.

\textbf{Actividad 2: Desarrollar modelo predictivo de Descripción}
\\
En esta actividad, se diseñó la arquitectura que se usó para la modalidad Descripción. Para ello, se realizó benchmarking sobre las propuestas de los autores enunciadas a continuación.

\begin{itemize}
	\item CNN: \cite{pr_cheng2019deeplearning}.
	\item LSTM: \cite{pr_jin2019dayssuccess}.
	\item Modelo Seq2seq: \cite{pr_lee2018contentDL}.
	\item Máquina de Vectores de Soporte o variantes: \cite{pr_sawhney2016usingLT}, \cite{pr_chen2019keywords_crowdfunding}.
	\item Regresión Logística: \cite{pr_mitra2014phrases}, \cite{pr_zhou2015projectdesc}.
	\item LDA o variantes: \cite{pr_yuan2016textanalytics}, \cite{pr_sawhney2016usingLT}.
\end{itemize}

\textbf{Entregable}: Red Neuronal Convolucional para modalidad Descripción.

\textbf{Actividad 3: Desarrollar modelo predictivo de Comentarios}
\\
En esta actividad, se diseñó la arquitectura que se usó para la modalidad Comentarios. Para ello, se realizó benchmarking sobre las propuestas de los autores enunciadas a continuación.

\begin{itemize}
	\item LSTM: \cite{pr_jin2019dayssuccess}, \cite{pr_shafqat2019topicpredictions}.
	\item LDA o variantes: \cite{pr_shafqat2019topicpredictions}.
	\item Modelo Seq2seq: \cite{pr_lee2018contentDL}, \cite{pr_jin2019dayssuccess}.
	\item HAN: \citeauthor{pr_lee2018contentDL}.
	\item Regresión Logística: \cite{pr_li2016predcrowd}, \cite{pr_kaur2017socmedcrowd}.
	\item Regresión Log-logística: \cite{pr_li2016predcrowd}.
\end{itemize}

\textbf{Entregable}: Modelo LSTM Bidireccional para modalidad Comentarios.

\textbf{Actividad 4: Desarrollar modelo predictivo ensamblado}
\\
En esta actividad, luego de implementar los modelos por cada modalidad, se ensamblaron 

\textbf{Entregable}:

\subsubsection{Evaluación}

\vspace{2ex}
%\begingroup
%\renewcommand\arraystretch{0.3}
\begin{longtable}{m{5cm}m{5cm}M{5cm}}
	\caption[Actividades de fase Evaluación]{Actividades de fase Evaluación.}
	\label{3:table7}
	\newcommand{\multirot}[1]{\multirow{2}{*}[-8ex]{\rotcell{\rlap{#1}}}}
	%\scriptsize
	\footnotesize
	%\centering
	\small
	%% Se agrega tabularnewline para longtable
	\tabularnewline \specialrule{.1em}{.05em}{.05em}
	\centering Actividades & \centering Descripción & Tareas
	\\
	\specialrule{.1em}{.05em}{.05em}
	Evaluar desempeño del modelo predictivo de Metainformación.
	& 
	&                                               
	\\
	\hline
	Evaluar desempeño del modelo predictivo de Descripción.
	& 
	& 
	\\
	\hline
	Evaluar desempeño del modelo predictivo de Comentarios.
	& 
	& 
	\\
	\hline
	Evaluar desempeño del modelo predictivo ensamblado.
	& 
	& 
	\\
	\specialrule{.1em}{.05em}{.05em}
\end{longtable}%
%\endgroup

%\par	%%Salto de linea
%\bigskip
\begin{flushleft}	%%Alinear a la izquierda sin justificar
	\small Fuente: Elaboración propia
\end{flushleft}



En el Capítulo V se explica cuáles fueron seleccionadas para evaluar cada modelo.

\textbf{Actividad 1: Evaluar desempeño del modelo predictivo de Metainformación}
\\


\textbf{Entregable}:

\textbf{Actividad 2: Evaluar desempeño del modelo predictivo de Descripción}
\\


\textbf{Entregable}:

\textbf{Actividad 3: Evaluar desempeño del modelo predictivo de Comentarios}
\\


\textbf{Entregable}:

\textbf{Actividad 4: Evaluar desempeño del modelo predictivo ensamblado}
\\


\textbf{Entregable}:

\subsubsection{Despliegue}

\vspace{2ex}
%\begingroup
%\renewcommand\arraystretch{0.3}
\begin{longtable}{m{5cm}m{5cm}M{5cm}}
	\caption[Actividades de fase Despliegue]{Actividades de fase Despliegue.}
	\label{3:table8}
	\newcommand{\multirot}[1]{\multirow{2}{*}[-8ex]{\rotcell{\rlap{#1}}}}
	%\scriptsize
	\footnotesize
	\centering
	\small
	%% Se agrega tabularnewline para longtable
	\tabularnewline \specialrule{.1em}{.05em}{.05em}
	\centering Actividades & \centering Descripción & Tareas
	\\
	\specialrule{.1em}{.05em}{.05em}
	Diseñar interfaz de captura de datos del prototipo.
	& 
	&                                              
	\\
	\hline
	Integrar modelo predictivo ensamblado al sistema.
	& 
	& 
	\\
	\hline
	Ejecutar prototipo con proyectos tecnológicos vigentes.
	& Ejecución del sistema prototipo en tiempo real con proyectos de tecnología de campañas vigentes.
	& 
	\\
	\specialrule{.1em}{.05em}{.05em}
\end{longtable}%
%\endgroup
%\par	%%Salto de linea
%\bigskip
\begin{flushleft}	%%Alinear a la izquierda sin justificar
	\small Fuente: Elaboración propia
\end{flushleft}

Luego de analizar el desempeño de cada modelo, tanto individual como el modelo final, se compararon los resultados con los antecedentes y se desarrolló una prueba piloto en la cual el modelo apilado entrenado fue ejecutado usando como entrada la URL de un proyecto aleatorio de Kickstarter para, luego de obtener sus características, predecir su estado de financiamiento. Esta acción también se encuentra detallada en el Capítulo V.

\textbf{Actividad 1: Diseñar interfaz de captura de datos del prototipo}
\\

\textbf{Entregable}:

\textbf{Actividad 2: Integrar modelo predictivo ensamblado al sistema}
\\

\textbf{Entregable}:

\textbf{Actividad 3: Ejecutar prototipo con proyectos tecnológicos vigentes}
\\

\textbf{Entregable}:

\subsection{Metodología para la medición de resultados}
Para seleccionar las métricas de clasificación, explicadas en la sección 2.2.9, que evaluarán a cada modelo por modalidad y al modelo final del trabajo, se realizó benchmarking sobre aquellas que fueron utizadas por los autores en los antecedentes de la investigación:

\begin{itemize}
	\item \textbf{Exactitud}: \cite{pr_chen2013kickpredict}, \cite{pr_chen2015predcrowd}, \cite{pr_beckwith2016predcrowd}, \cite{pr_yuan2016textanalytics}, \cite{pr_sawhney2016usingLT}, \cite{pr_kaur2017socmedcrowd}, \cite{pr_kamath2018suplearn}, \cite{pr_yu2018deeplearning}, \cite{pr_lee2018contentDL}, \cite{pr_cheng2019deeplearning}, \cite{pr_chen2019keywords_crowdfunding}, \cite{pr_shafqat2019topicpredictions}.
	\item \textbf{Precisión}: \cite{pr_beckwith2016predcrowd}, \cite{pr_yuan2016textanalytics}, \cite{pr_kaur2017socmedcrowd}, \cite{pr_cheng2019deeplearning}.
	\item \textbf{Sensibilidad}: \cite{pr_beckwith2016predcrowd}, \cite{pr_yuan2016textanalytics}, \cite{pr_kaur2017socmedcrowd}, \cite{pr_cheng2019deeplearning}, \cite{pr_chen2019keywords_crowdfunding}.
	\item \textbf{Especificidad}: \cite{pr_chen2019keywords_crowdfunding}.
	\item \textbf{Ratio de Falsa Alarma}: \cite{pr_kaur2017socmedcrowd}.
	\item \textbf{Media Geométrica (G-Mean)}: \cite{pr_chen2019keywords_crowdfunding}.
	\item \textbf{Puntaje F1}: \cite{pr_zhou2015projectdesc}, \cite{pr_beckwith2016predcrowd}, \cite{pr_yuan2016textanalytics}, \cite{pr_kaur2017socmedcrowd}, \cite{pr_cheng2019deeplearning}, \cite{pr_chen2019keywords_crowdfunding}.
	\item \textbf{Curva ROC}: \cite{pr_zhou2015projectdesc}, \cite{pr_beckwith2016predcrowd}.
	\item \textbf{Área bajo la Curva ROC (AUC)}: \cite{pr_beckwith2016predcrowd}, \cite{pr_li2016predcrowd}, \cite{pr_kaur2017socmedcrowd}, \cite{pr_yu2018deeplearning}, \cite{pr_cheng2019deeplearning}.
	\item \textbf{Área bajo la Curva Precisión-Sensibilidad (PRC)}: \cite{pr_kaur2017socmedcrowd}.
	\item \textbf{Error de Validación Cruzada}: \cite{pr_mitra2014phrases}.
	\item \textbf{Coeficiente de Correlación de Matthew (MCC)}: \cite{pr_kaur2017socmedcrowd}.
	\item \textbf{Divergencia de Kullback-Leibler (KL)}: \cite{pr_jin2019dayssuccess}.
	\item \textbf{Raíz del Error Cuadrático Medio (RMSE)}: \cite{pr_jin2019dayssuccess}.
	\item \textbf{Error Absoluto Medio (MAE)}: \cite{pr_jin2019dayssuccess}.
	\item \textbf{Índice de Concordancia (CI)}: \cite{pr_jin2019dayssuccess}.
\end{itemize}

\section{Cronograma de actividades y presupuesto}
Se elaboró un cronograma de actividades de toda la investigación, mostrada en la Figura \ref{3:fig10}, contemplando desde el inicio de la misma, desarrollo, evaluación de resultados y sustentación.

\begin{figure}[h]
	\centering
	\small
	\begin{subfigure}{1\textwidth}
		\centering
		\includegraphics[width=1\linewidth]{3/figures/cronograma_2020.png}
		\caption{Año 2020}
	\end{subfigure}

	\begin{subfigure}{1\textwidth}
		\centering
		\includegraphics[width=0.9\linewidth]{3/figures/cronograma_2021.png}
		\caption{Año 2021}
	\end{subfigure}

	\caption[Cronograma de actividades de la investigación]{Cronograma de actividades de la investigación.\\
	Fuente: Elaboración propia.}
	\label{3:fig10}
\end{figure}

La partida presupuestal de todo el proyecto se divide en dos partes: los costos personales del autor y los costos de las herramientas para la operación del proyecto.

Los costos personales del autor del trabajo de tesis se muestran en la Tabla \ref{3:table9}. Estos incluyen las herramientas adquiridas antes del inicio de la investigación como la laptop, así como los pagos de servicios generales y del trámite de elaboración y sustentación pública de Tesis. Se menciona la cantidad de horas utilizadas por el autor para el desarrollo del trabajo; sin embargo no se considera un valor monetario por cada unidad al ser un valor incalculable.

\begin{table}[h!]
	\caption[Presupuesto de los costos personales del autor]{Presupuesto de los costos personales del autor.}
	\label{3:table9}
	\centering
	\small
	\begin{tabular}{lcrr}
		\specialrule{.1em}{.05em}{.05em}
		\multicolumn{1}{c}{\centering{Item}} & \multicolumn{1}{c}{\centering{Tiempo usado (horas)}} & \multicolumn{1}{c}{\centering{Costo (soles)}} & \multicolumn{1}{c}{\centering{Subtotal}}
		\\
		\specialrule{.1em}{.05em}{.05em}
		\multicolumn{4}{l}{Recursos materiales}
		\\
		Laptop Lenovo ideapad 330 Core i7 8va Gen  &  & S/.4,500.00 & S/.4,500.00
		\\
		\hline
		\multicolumn{4}{l}{Pagos del trámite de elaboración y sustentación pública de Tesis}
		\\
		Derecho de inscripción de tema de investigación &  & S/.800.00 & S/.800.00
		\\
		Reserva del tema de tesis  &  & S/.2,700.00 & S/.2,700.00 \\
		Derecho de sustentación                                                          & & S/.1,500.00                                                                               & S/.1,500.00                                                                              \\
		\hline
		\multicolumn{4}{l}{Recursos humanos}
		\\
		Avance de tesis                                                                  & \multicolumn{1}{r}{900}                                                                    & Incalculable                                                                              & -                                                                                        \\
		\hline
		\multicolumn{4}{l}{Servicios generales}
		\\
		Internet + luz (7 meses)                                                         & \multicolumn{1}{r}{110}                                                                    & S/.80.00                                                                                  & S/.560.00                                                                              \\
		\specialrule{.1em}{.05em}{.05em}
		Total &  &  & \multicolumn{1}{l}{S/.10,060.00}
		\\
		\specialrule{.1em}{.05em}{.05em}
	\end{tabular}
	\par	%%Salto de linea
	\bigskip
	\begin{flushleft}	%%Alinear a la izquierda sin justificar
		\small Fuente: Elaboración propia.
	\end{flushleft}
\end{table}

Asimismo, también se contemplan los costos por uso de servicios en la nube como parte de la extracción de comentarios desde servidores en Google Cloud Platform (GCP) y desarrollo de modelos en Google Colab Pro en la Tabla \ref{3:table10}.

\begin{table}[h!]
	\caption[Presupuesto de los costos de las herramientas para el proyecto]{Presupuesto de los costos de las herramientas para el proyecto.}
	\label{3:table10}
	\centering
	\small
	\begin{tabular}{lcrcr}
		\specialrule{.1em}{.05em}{.05em}
		\multicolumn{1}{c}{\centering{Item}} & \multicolumn{1}{c}{\centering{Unidades}} & \multicolumn{1}{c}{\centering{Costo (dólares)}} & \multicolumn{1}{l}{\centering{Horas}} & \multicolumn{1}{c}{\centering{Subtotal}} \\
		\specialrule{.1em}{.05em}{.05em}
		\multicolumn{4}{l}{Google Cloud Platform} & -\$98.56
		\\
		Instancia VM Ubuntu (g1-small, 12GB RAM) & 1 & \$0.021 & 310 & \$6.51 \\
		Imagen de instancia Ubuntu & 8 & \$0.02 & 306 & \$48.96                                                                              \\
		Instancia VM de Windows Server 2019 (4GB RAM) & 1 & \$0.086 & 300 & \$25.80                                                                              \\
		Costo por instancias encendidas & 10 & & & \$3.41                                                                               \\
		Uso de instancias posteriormente eliminadas & & & & \$95.88                                                                              \\
		Pago mensual (luego de aceptar mejora de plan) & 4 & \$5.22 & & \$20.88                                                                              \\
		Crédito de \$300.00 por 12 meses & 1 & -\$300.00 & & -\$300.00                                                                            \\
		\hline
		\multicolumn{4}{l}{Google Colab Pro} & \$49.95
		\\
		Pago mensual (luego de aceptar mejora de plan) & 5 & \$9.99 & \multicolumn{1}{l}{} & \$49.95                                                                              \\
		\specialrule{.1em}{.05em}{.05em}
		Total a pagar &  &  &  & \$49.95
		\\
		\specialrule{.1em}{.05em}{.05em}
	\end{tabular}
	\par	%%Salto de linea
	\bigskip
	\begin{flushleft}	%%Alinear a la izquierda sin justificar
		\small Fuente: Elaboración propia.
	\end{flushleft}
\end{table}

Los montos totales por el uso de instancias creadas en GCP son descontados de los \$300.00 en crédito gratuito válidos por 12 meses (a la fecha en que se desarrolló la investigación) \parencite{ot_googlecloud_freetrial}, recibidos inicialmente desde el 12 de agosto del 2020 (fecha de inscripción) para poder ser usados como versión de prueba. A partir del siguiente mes, se mejoró el plan para habilitar más servicios (por ejemplo, máquina virtual en Windows) y comenzó a facturarse \$5.22 mensualmente.