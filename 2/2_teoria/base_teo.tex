\subsection{Machine Learning}
 Es un subcampo de l]ecutar dificultosos procesos aprendiendo de datos, en lugar de seguir reglas preprogramadas \parencite{tec_royal2017machine}.

 es importante mencionar que existen también cinco tipos de problemas de aprendizaje que se pueden enfrentar: regresión, clasificación, simulación, optimización y clusterización \parencite{bk_gollapudi2016practical}. Por otro lado, el aprendizaje automático también posee una división por subcampos que se puede observar en la Figura 14.
%%Figura
 \subsection{Natural Language Processing (NLP)}
 Naturalmano \parencite{bk_goyal2018deep}. Otra definición para este término implica que es un campo especializado de la informática que es

 De acuerdo con \citet{bk_goyal2018deep}, e
 