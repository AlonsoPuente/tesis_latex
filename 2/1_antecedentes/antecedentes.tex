En esta sección se presentarán diversos trabajos de investigación basados en la predicción de éxito o fracaso de campañas en Kickstarter o plataformas similares y el análisis de estas utilizando conjunto de datos de la propia plataforma o almacenadas en otros repositorios, con sus variables respectivas. En la mayoría de casos consideraron variables básicas que se obtienen del repositorio de las plataformas de crowdfunding, en otros casos consideraron variables cualitativas basadas en texto y descripción de proyectos, y en otros antecedentes usaron nuevas técnicas poco convencionales como el Aprendizaje Profundo e híbridos de modelos para obtener los mejores resultados posibles.
Asimismo, a continuación se presenta un cuadro resumen (véase Anexo \ref{anexo4}) de lo que se presenta en esta sección.

\subsection{Primer antecedente: «Supervised Learning Model For Kickstarter Campaigns With R Mining» \citep*{pr_kamath2018suplearn}}
\citeauthor{pr_kamath2018suplearn} realizaron un artículo de investigación el cual fue publicado en la revista «International Journal of Information Technology, Modeling and Computing» en el año 2018. Este fue titulado \citetitle{pr_kamath2018suplearn} la cual traducida al español significa «Modelo de aprendizaje supervisado para campañas de Kickstarter con R Mining».

\subsubsection{Planteamiento del Problema y objetivo}
Kickstarter actualmente es la plataforma web basado en crowdfunding más grande existente. Sin embargo, no todas las campañas que se encuentran en su portal son exitosas. Algunos trabajos previos citados en este artículo citan que el lenguaje usado en las campañas alcanza el poder predictivo de 58.86\%, es decir, la descripción semántica de un proyecto ayuda considerablemente en la predicción de éxito debido a que muchos patrocinadores se fijan en la calidad presente en una campaña antes de invertir.  Sin embargo, es dejada de lado en muchos trabajos de investigación destinados a la predicción de éxito de financiamiento, así como también las interacciones de un proyecto en redes sociales. Para el tiempo en que este artículo fue publicado, el nivel de exactitud de los modelos predictivos ya existentes bordeaba por el 68\% como el descrito anteriormente

Ante esto, autores desarrollaron un sistema con técnicas de Aprendizaje Automático usando el entorno R aplicada a conjunto de datos de campañas en Kickstarter para clasificar proyectos entrenando diferentes clasificadores en esta data y predecir con un alto nivel de exactitud.

\subsubsection{Técnicas empleadas por los autores}
Los autores elaboraron 5 modelos clasificadores para la investigación: Árbol de decisión, Bosque Aleatorio, Red Neuronal Artificial, Vecino K más cercano, y Naïve Bayes. Se consideraron como variables para estos modelos al nombre del proyecto, URL del proyecto, descripción del proyecto, categoría, sub-categoría, número de patrocinadores, monto total patrocinado, meta de financiación, fecha de lanzamiento del proyecto, duración del proyecto, número de actualizaciones, número de recompensas, si el proyecto cuenta con video, localización del proyecto, y creador del proyecto.

\subsubsection{Metodología empleada por los autores}
En primer lugar, se analizó y definió el problema. A continuación, se recolectaron 120 proyectos de Kickstarter a través de su URL. Luego, las variables categóricas fueron transformadas en numéricas como parte del pre-procesamiento. Esto permitió realizar extracción y exploración del conjunto final mediante estadísticas descriptivas con paquetes de R. Una vez analizado los datos, se procedió a elaborar los modelos clasificadores y finalmente, con la matriz de confusión, se seleccionó el mejor de ellos evaluándolos mediante la exactitud.

\subsubsection{Resultados obtenidos}
Bajo la métrica de exactitud, el mejor de los 5 modelos clasificadores construidos fue la Red Neuronal, con un valor de 0.94.


\subsection{Segundo antecedente: «Predicting Success in Equity Crowdfunding» \citep*{pr_beckwith2016predcrowd}}
\citeauthor{pr_beckwith2016predcrowd} realizó un artículo de investigación el cual fue publicado en la revista «Joseph Wharton Scholars» de la Universidad de Pensilvania en el año 2016. Este fue titulado \citetitle{pr_beckwith2016predcrowd} la cual traducida al español significa «Prediciendo el éxito en el financiamiento colectivo de acciones».

\subsubsection{Planteamiento del Problema y objetivo}
El financiamiento colectivo o crowdfunding de inversión está haciéndose cada vez más popular. Este concepto permite que los emprendedores ofrezcan algún tipo de producto o servicio como compensación por contribuciones financieras una vez que ya se tengan ventas reales o acuerdos comerciales, a diferencia por ejemplo del crowdfunding de recompensas que ofrece algo a sus patrocinadores desde la concepción del proyecto. Resulta ser muy interesante para los patrocinadores ya que no se necesita contar con un capital muy alto. A cambio, ellos recibirán algo a cambio una vez que el proyecto se realice. Este factor permite su mayor probabilidad de éxito de financiamiento al momento de darse una campaña de este tipo de crowdfunding ya que los patrocinadores pueden invertir en más de un proyecto a la vez. A partir de este punto nace la siguiente cuestión: “Dados diferentes start-ups con similares características observables, ¿qué motiva a pequeños patrocinadores a invertir en ciertos start-ups y no en otros?”.

Por ello, el objetivo del autor fue determinar la relación entre las características de una compañía determinada y su capacidad para recaudar fondos en la plataforma de financiamiento colectivo de capital AngelList.

\subsubsection{Técnicas empleadas por los autores}
El autor desarrolló un modelo de Regresión logística usando las siguientes variables: si la empresa previamente había recibido o no financiación, si la compañía cuenta con perfil en Twitter, número de veces que la compañía había sido mencionada en alguna publicación, número de veces que la compañía había sido mencionada en TechCrunch, número de personas listadas como co-fundadoras en el perfil de la compañía AngelList, si AngelList lista entre 11 y 50 empleados como tamaño de la empresa, si la compañía está ubicada en San Francisco, si entre los fundadores de AngelList al menos uno de ellos cuenta con un MBA, si entre los fundadores de AngelList al menos uno de ellos realizó sus estudios en alguna de las mejores 20 universidades de Estados Unidos, y el número de cierre del S\&P 500 el día anterior al lanzamiento de la campaña de crowdfunding de AngelList.
Este modelo se comparó con un Árbol de Decisión CART, un modelo de Naïve Bayes y una Máquina de Vectores de Soporte.

\subsubsection{Metodología empleada por los autores}
Inicialmente, se recolectaron más de 5 mil compañías de AngelList que solicitaron contribuciones financieras. De esta cantidad, se considerarons aquellas con datos completos (2,603 empresas). Luego del pre-procesamiento del conjunto final de datos, se elaboró el modelo y se evaluó el modelo mediante la exactitud, precisión, sensibilidad, puntaje F1, y área bajo la curva (AUC).

\subsubsection{Resultados obtenidos}
Bajo las métricas mencionadas, el modelo propuesto por los autores superó a los otros 3 modelos comparados en cada una, tomando como resultados 0.87 de exactitud, precisión promedio de 0.85, sensibilidad promedio de 0.88, puntaje F1 promedio de 0.86, y área bajo la curva de 0.74.


\subsection{Tercer antecedente: «Money Talks: A Predictive Model on Crowdfunding Success Using Project Description» \citep*{pr_zhou2018projectdesc}}
\citeauthor{pr_zhou2018projectdesc} realizaron un resumen de la conferencia «Twenty-first Americas Conference on Information Systems» en Puerto Rico en el año 2018. Este fue titulado \citetitle{pr_zhou2018projectdesc} la cual traducida al español significa «Money Talks: un modelo predictivo sobre el éxito del crowdfunding utilizando la descripción del proyecto».

\subsubsection{Planteamiento del Problema y objetivo}
Las investigaciones existentes de crowdfunding se centran principalmente en las variables básicas del proyecto como la categoría y la meta; sin embargo, son muy pocos estudios basados en el contenido de la información, es decir, en la descripción del mismo.

El objetivo de los autores fue estudiar la influencia y el impacto del uso de descripciones de proyectos en su éxito de financiación para predecirlo.

\subsubsection{Técnicas empleadas por los autores}
Se desarrolló un modelo de Regresión Logística usando variables previamente identificadas (meta del proyecto, duración del proyecto, si el creador tiene conección a Facebook, número de amigos en Facebook del creador, si el proyecto incluye una imagen, si el proyecto incluye un video, número de recompensas, año de lanzamiento del proyecto, categoría del proyecto) y nuevas introducidas (número de palabras en la descripción del proyecto, legibilidad de descripción del proyecto medido por Índice de niebla Gunning, ratio de positivo y negativo en descripción del proyecto, número de proyectos previamente creados y número de proyectos previamente patrocinados por el autor).

\subsubsection{Metodología empleada por los autores}
Se comenzó con la operacionalización de los constructos, es decir, cómo se consideraría la información recolectada a partir de los antecedentes en la literatura. El siguiente paso fue la recolección de información de más de 154 mil proyectos de Kickstarter entre 2009 y 2014. Y finalmente, se construyó el modelo predictivo y se comparó contra otros modelos de la base y convencionales.

\subsubsection{Resultados obtenidos}
Todos los modelos (base, convencional y propuesto) fueron evaluados por el valor-F probando en validaciones cruzadas de 3, 5 y 10 iteraciones. Bajo esta métrica, el mejor modelo fue el propuesto, alcanzando un valor-F de 71.17 con 10 iteraciones.


\subsection{Cuarto antecedente: «The Determinants of Crowdfunding Success: A Semantic Text Analytics Approach» \citep*{pr_yuan2016textanalytics}}
\citeauthor{pr_yuan2016textanalytics} realizaron un artículo de investigación el cual fue publicado en la revista «Decision Support Systems» en el año 2016. Este fue titulado \citetitle{pr_yuan2016textanalytics} la cual traducida al español significa «Los determinantes del éxito del financiamiento colectivo: Un enfoque de analítica semántica de texto».

\subsubsection{Planteamiento del Problema y objetivo}
Actualmente existen diversos estudios de éxito de campañas de financiamiento colectivo, la mayoría usando modelos predictivos en proyectos crowdfunding basados en recompensas y análisis de características poco profundas del lenguaje en descripción de los proyectos (por ejemplo, número de palabras, errores ortográficos, etc) así como en la implementación de métodos de Aprendizaje Automático.

Como propuesta alterna, los autores construyeron un marco de trabajo basado en análisis de texto que pueda extraer semánticas de descripciones textuales de proyectos para predecir sus resultados de recaudación de fondos.

\subsubsection{Técnicas empleadas por los autores}
Se creó un modelo de Bosque Aleatorio para los datos numéricos, y POS Tag Stemming más modelo DC-LDA más características tópicas para los datos textuales. Esta arquitectura se comparó con otros modelos en la literatura de la publicación.

\subsubsection{Metodología empleada por los autores}
Se recolectaron 500 proyectos de 2 webs chinas sobre crowdfunding cada una. Luego, se construyó el framework de análisis textual. A continuación, se desarrolló un modelo DC-LDA para una extracción más efectiva de las características tópicas a partir de las descripciones. Y finalmente, se realizó un análisis empírico para identificar características discriminatorias que influye en el éxito de la recaudación de fondos.

\subsubsection{Resultados obtenidos}
Para los datos numéricos, la mejor performance se obtuvo evaluando el modelo con la sensibilidad para el Random Forest (0.98 en la segunda data); mientras que para los datos textuales, el modelo DC-LDA (al ser comparado con el LDA normal) obtuvo un puntaje F1 de 0.91.


\subsection{Quinto antecedente: «Will your Project get the Green light? Predicting the success of crowdfunding campaigns» \citep*{pr_chen2015predcrowd}}
\citeauthor{pr_chen2015predcrowd} realizaron un resumen de la conferencia «Pacific Asia Conference on Information Systems (PACIS) 2015» en New York, Estados Unidos, en el año 2015. Este fue titulado \citetitle{pr_chen2015predcrowd} la cual traducida al español significa «¿Tu proyecto obtendrá la luz verda? Prediciendo el éxito de campañas de financiamiento colectivo».

\subsubsection{Planteamiento del Problema y objetivo}
Desde el surgimiento de campañas de financiamiento colectivo en sitios web así como estudios de estos casos, la mayoría de estos presentan problemas en la predicción de éxito de una campaña por distintas casuísticas ya que suelen enfocarse en aquellos concluídos. ¿Será posible desarrollar una técnica efectiva para predecir si una campaña será exitosa o no en diferentes momentos de tiempo de las campañas de crowdfunding?

Para responder esta interrogante, los autores desarrollaron una técnica efectiva para predecir si una campaña de crowdfunding tendrá éxito o fracasará a través de extracción y posterior uso de características estáticas y dinámicas.

\subsubsection{Técnicas empleadas por los autores}
Los autores construyeron una serie de 8 modelos de Bosque Aleatorio para diferentes etapas (puntos de tiempo) de la campaña desde el día 0 hasta el día 7.

\subsubsection{Metodología empleada por los autores}
Se recolectaron más de 4 mil proyectos de Kickstarter obtenido directamente desde el sitio, así como con ayuda de la página web Kickspy, entre el primer y último día de abril del 2014. Una vez pre-procesada la data, se extrajeron las características para la tarea de predicción de objetivos. Luego, estas características se clasificaron en 5 categorías: características intrínsecas, mecanismo financiero, calidad y sentimiento del contenido, interacción social y efecto de progresión, en donde las 3 primeras + interacción social corresponden a un nuevo grupo asignado como «características estáticas», mientras que la última + interacción social se asignó como «características dinámicas». Ambos nuevos conjuntos agrupados finalmente fueron entrenados con la serie de modelos para diferentes etapas.

\subsubsection{Resultados obtenidos}
Se comparó la serie de modelos propuestos considerando todas las características con modelos que solo consideraban algunas, así como también comparado con uno de la literatura. El resultado final de la evaluación medidos por la exactitud determinó que la primera opción mencionada logró la mejor performance, con un valor de 0.8467.


\subsection{Sexto antecedente: «Project Success Prediction in Crowdfunding Environments» \citep*{pr_li2016predcrowd}}
\citeauthor{pr_li2016predcrowd} realizaron un resumen de la conferencia «The Ninth ACM International Conference on Web Search and Data Mining (WSDM’ 16)» en San Francisco, Estados Unidos, en el año 2016. Este fue titulado \citetitle{pr_li2016predcrowd} la cual traducida al español significa «Predicción de éxito de un proyecto en ambientes de financiamiento colectivo».

\subsubsection{Planteamiento del Problema y objetivo}
Durante la última década se han elaborado una serie de modelos de clasificación que permitan pronosticar con cierto nivel de exactitud si algunos proyectos de financiamiento colectivo tendrán éxito o no. Sin embargo, el hecho de estimar si esto ocurrirá durante el plazo dado de la campaña no puede proporcionar una guía adecuada a los patrocinadores que desean invertir en proyectos populares. Es entonces que se cuestiona la posibilidad de predecir el éxito de campañas en sitios web de crowdfunding como Kickstarter considerando características obtenidas durante la operación de la campaña.

Para ello, los autores formularon la predicción del éxito del proyecto como un problema de análisis de supervivencia y aplicar el enfoque de regresión censurada.

\subsubsection{Técnicas empleadas por los autores}
Los autores desarrollaron 2 modelos para evaluar la regresión censurada: 1 modelo de distribución logística y 1 de distribución log-logística. Para comparar con sus modelos, también se elaboraron otros métodos para manejar observaciones censuradas, mencionados en la literatura como Cox, Regresión Tobit, estimación Buckley-James e Índice de Concordancia Boosting (BoostCI).

\subsubsection{Metodología empleada por los autores}
Inicialmente, se recolectaron más de 27 mil proyectos de Kickstarter entre diciembre del 2013 y junio del 2014 para la data estática, a los cuales se removieron los cancelados, suspendidos o con menos de 1 patrocinador y monto prometido de \$100. Asimismo, para la data dinámica se capturaron más de 106 mil tweets de Twitter que contenían el enlace rápido hacia la página de Kickstarter. Una vez obtenidos los conjuntos de datos, se pre-procesaron para finalmente elaborar los modelos predictivos.

\subsubsection{Resultados obtenidos}
Como resultados de las múltiples pruebas en los modelos, en los cuales los autores experimentaron distintas combinatorias con la data estática y la dinámica, así como agregando y desagregando proyectos fracasados, se obtuvo que el modelo de Regresión log-logística evaluado con el área bajo la curva (AUC) presentó mejor performance que el resto en 3 de las 4 combinatorias, alcanzando su mejor nivel en el conjunto de datos que combina data estática, dinámica, de características de proyectos transcurridos los primeros 3 días y con proyectos fracasados, con un puntaje Survival AUC de 0.9030.


\subsection{Séptimo antecedente: «Effect of Social Media Connectivity on Success of Crowdfunding Campaigns» \citep*{pr_kaur2017socmedcrowd}}
\citeauthor{pr_kaur2017socmedcrowd} realizaron un artículo de investigación el cual fue publicado en la revista «Procedia Computer Science» en el año 2017. Este fue titulado \citetitle{pr_kaur2017socmedcrowd} la cual traducida al español significa «Efecto de conectividad en medios sociales en éxito de campañas de financiamiento colectivo».

\subsubsection{Planteamiento del Problema y objetivo}
Los proyectos de crowdfunding, así como los proyectos lanzados en sitios dedicados a esto, han crecido exponencialmente en los últimos años. Los medios sociales desempeñan un papel importante en la difusión de palabras sobre una campaña y en la obtención de fondos con éxito. Después de lanzarse una campaña, su éxito se define por la interacción social de los creadores en la plataforma y en las redes sociales. Además, el tamaño de la red social del creador y su presencia en línea motiva a los patrocinadores a participar y financiar el proyecto. Partiendo de esta premisa, se busca conocer el impacto de la conectividad con las redes sociales y las interacciones sociales en el rendimiento del crowdfunding.

Para ello, los autores plantearon el desarrollo de un modelo predictivo que comprenda las características de la campaña y sumado a eso, la información sobre la conectividad con las diversas redes sociales como Facebook y Twitter.

\subsubsection{Técnicas empleadas por los autores}
Los autores propusieron un modelo de Regresión Logística, el cual compararon con algunos enunciados en sus antecedentes como Naïve Bayes, J48 y Bosques Aleatorios.

\subsubsection{Metodología empleada por los autores}
Se recolectaron 2 tipos de conjuntos de datos: el primero obtenido desde Kickstarter en el mes de Abril del 2014, con más de 4 mil campañas; mientras que el segundo a través del perfil del creador vinculado a redes sociales. El primer conjunto comprendió las variables de categoría, tipo de divisa, monto de la meta, monto prometido, número de recompensas, duración de la campaña en días, y el estado final de financiamiento (exitoso o fracasado). Por el lado del segundo conjunto, además de obtener marcas de hacia qué redes sociales el creador se conecta, mediante el enlace se extrajeron más de 19 mil tweets en el que se promocionaron las campañas gracias a la API de Twitter.

Una vez armados los conjuntos de datos, se creó el modelo propuesto y se comparó con las bases de referencia usando como métrica principal la exactitud. Previo a este paso, todas las variables consideradas se evaluaron en una matriz de correlaciones con el valor de “exitoso” de la variable dependiente.

\subsubsection{Resultados obtenidos}
Culminados los experimentos de la selección de variables en la matriz de correlaciones, así como la ejecución del modelo de Regresión Logística, la exactitud de este alcanzó un valor de 0.767 para un subconjunto de 66\% de entrenamiento y de 34\% de prueba, indicando así que la conectividad a medios sociales e interacciones en ellos influye en la predicción de éxito de la campaña. Asimismo, evaluando también con precisión y sensibilidad promedio, el ratio alcanzó 0.768 y 0.767 respectivamente.


\subsection{Octavo antecedente: «Prediction of Crowdfunding Project Success with Deep Learning» \citep*{pr_yu2018deeplearning}}
\citeauthor{pr_yu2018deeplearning} realizaron un resumen de la conferencia «2018 IEEE 15th International Conference on e-Business Engineering (ICEBE)» del 12 al 14 de octubre del 2018 en Xi'an, China. Este fue titulado \citetitle{pr_yu2018deeplearning}, la cual traducida al español significa «Predicción del éxito de proyecto de financiamiento colectivo con Aprendizaje Profundo».

\subsubsection{Planteamiento del Problema y objetivo}
Debido al aumento dramático de la actividad de financiamiento colectivo en los últimos años, en el cual tanto creadores como patrocinadores participan, son más las campañas que buscan alcanzar su objetivo. Sin embargo, según el análisis empírico, solo la tercera parte lo logra. La incógnita que rodea a esta situación es la posibilidad de elaborar un modelo predictivo de éxito en proyectos de Kickstarter utilizando distintas técnicas al Aprendizaje Automático.

Por ello, los autores plantean como objetivo el desarrollo de un modelo que prediga el éxito del proyecto de crowdfunding con Aprendizaje Profundo usando registros históricos de campañas en Kickstarter.

\subsubsection{Técnicas empleadas por los autores}
Los autores elaboraron un modelo basado en un Perceptrón Multicapa (MLP), alimentado por 6 variables visibles de la campaña (metadata). Además, sus resultados se compararon con otros modelos considerados en sus antecentes como Bosques aleatorios, AdaBoost, Máquina de vectores de soporte, Árboles de decisión, Regresión logística y Naïve Bayes.

\subsubsection{Metodología empleada por los autores}
Se recolectaron más de 378 mil campañas de Kickstarter entre Mayo 2009 y Marzo 2018, recopilados retrospectivamente desde Kaggle. A continuación, realizaron análisis exploratorio de los datos para entender la data y seleccionar variables, y finalmente construyeron un modelo, el cual se comparó sus resultados con los de otros modelos.

\subsubsection{Resultados obtenidos}
Para comparar los resultados del modelo propuesto con los antecedentes, se tomaron en cuenta como métricas la exactitud y el área bajo la curva (AUC). De acuerdo a ellas, para ambos escenarios se lograron mejor performance, con una exactitud de 0.9320 y un área bajo la curva de 0.9323 frente al mejor modelo de los antecedentes, Bosques Aleatorios, que obtuvo como resultados 0.9293 y 0.9272 para las mismas métricas respectivamente.

\subsection{Noveno antecedente: «Estimating the Days to Success of Campaigns in Crowdfunding: A Deep Survival Perspective» \citep*{pr_jin2019dayssuccess}}
\citeauthor{pr_jin2019dayssuccess} realizaron un resumen de la conferencia «The 33rd AAAI Conference on Artificial Intelligence (AAAI'2019)» del 27 de enero al 1 de febrero del 2019 en Honolulu, Hawai. Este fue titulado \citetitle{pr_jin2019dayssuccess} la cual traducida al español significa «Estimación de días de éxito de campañas en financiamiento colectivo: Una perspectiva de supervivencia profunda».

\subsubsection{Planteamiento del Problema y objetivo}
Dado el incremento de actividad en campañas de crowdfunding en sitios como Kickstarter o Indiegogo, los estudios sobre este tema también han ido de la mano. Sin embargo, la mayoría de investigaciones se enfocan en predecir el ratio de éxito de financiamiento en una campaña. El reto de predecir el tiempo de duración es más complicado porque se ve afectada por la variabilidad de la metadata y los comentarios del proyecto, resultando un alto nivel de fracaso (60\%) en la financiación de proyectos.

Los autores plantean la predicción de la fecha exacta de finalización de una campaña, así como controlar la distribución de patrocinios conforme a la evolución del tiempo.

\subsubsection{Técnicas empleadas por los autores}
Se elaboró un modelo Seq2seq (encoder + decoder) con previos multifacéticos (SMP) para la distribución de patrocinios y el tiempo de éxito, así como también un modelo evolutivo lineal para mantener el cambio de las distribuciones de patrocinios.

\subsubsection{Metodología empleada por los autores}
Se recolectó más de 14 mil proyectos de Indiegogo con información estática (descripción de campaña, descripción de las recompensas, categoría de la campaña, tipo del creador, duración declarada de financiación, meta declarada prometida, número de recompensas, precio máximo/mínimo/promedio de recompensas, plazo máximo/mínimo/promedio de entrega) y dinámica (comentarios), luego se definió el problema estudiado. A continuación, se introdujo los detalles técnicos de SMP, el conjunto pre-procesado de entrada, la arquitectura SMP, los previos multifacéticos y por último, el método de entrenamiento del modelo.

\subsubsection{Resultados obtenidos}
Respecto a la predicción de distribución de patrocinios, y evaluando mejor con RECM, los modelos propuestos de SMP fueron los más efectivos (0.135 para 70\% de ratio de partición) para el Encoder.
Respecto a la predicción del tiempo de éxito, los modelos propuestos de SMP (el mejor con 0.96 para 50\% de ratio de partición) fueron mejores que modelos de referencia para el Decoder.

\subsection{Décimo antecedente: «Success Prediction on Crowdfunding with Multimodal Deep Learning» \citep*{pr_cheng2019deeplearning}}
\citeauthor{pr_cheng2019deeplearning} realizaron un resumen de la conferencia «The Twenty-Eighth International Joint Conference on Artificial Intelligence (IJCAI-19)» del 10 al 16 de agosto del 2019 en Macao, China. Este fue titulado \citetitle{pr_cheng2019deeplearning} la cual traducida al español significa «Predicción del éxito en financiamiento colectivo con Aprendizaje Profundo Multimodal».

\subsubsection{Planteamiento del Problema y objetivo}
La mayoría de los enfoques de predicción existentes aprovechan solo la modalidad dominada por el texto a pesar de existir más información en los perfiles de los proyectos, como por ejemplo, metadatos e imágenes. A estos últimos se han realizado poco trabajo para evaluar sus efectos hacia la predicción del éxito. Además, la metainformación ha sido explotada en muchos enfoques existentes para mejorar la precisión de la predicción. Sin embargo, esta generalmente se limita a la dinámica después de la publicación de los proyectos, haciendo que tanto los creadores del proyecto como las plataformas no puedan predecir el resultado de manera oportuna. Se carecen estudios basados en distintas modalidades más allá de la metainformación.

El objetivo de los autores es construir esquemas avanzados de redes neuronales que combinan información de diferentes modalidades para predecir el estado de financiamiento del proyecto usando solo información pre-publicación.

\subsubsection{Técnicas empleadas por los autores}
Debido a que los autores construyeron un modelo apilado de 3 modelos para cada modalidad, cada una se distribuyó de la siguiente manera: un modelo de Máquina de Vectores de Soporte (SVM) para la metainformación; la Red Neuronal Convolucional (CNN) pre-entrenada de 16 capas VGG16, trabajando junto con la Bolsa de Palabras Visuales (BoVW) para las imágenes; y la Bolsa de Palabras (BoW) trabajando junto con el método Frecuencia de Términos - Frecuencia Inversa de Documentos (TF-IDF), con modelo de incrustaciones de palabras GloVe de 300 dimensiones.

\subsubsection{Metodología empleada por los autores}
Se recolectaron más de 20 mil campañas finalizadas (exitosas o fracasadas) en Kickstarter gracias a algoritmo creado para captura de datos en la página por 2 semanas. Se descartaron proyectos previos al 2015, resultando así proyectos entre este año y el 2018. Asimismo, entre el contenido obtenido se registraron perfil textual (título, resumen, descripción del financiamiento, y riesgos/retos), imágenes (en formato jpg), categorías, meta de financiamiento, fecha de inicio y de finalización de campaña. Luego, se realizó pre-procesamiento a los subconjuntos de datos y se dividieron en 3 subconjuntos: entrenamiento (proyectos del 2015 y 2016), validación (2017) y prueba (2018). A continuación, se configuró la arquitectura del marco de trabajo del modelo Aprendizaje Profundo Multimodal y finalmente se realizaron distintos experimentos elaborando combinatorias de distintas modalidades para compararlas entre sí mediante las métricas de sensibilidad, precisión, puntaje F1 y área bajo la curva (AUC).

\subsubsection{Resultados obtenidos}
Realizando una comparación entre la base de referencia de un modelo SVM con texto, imágenes y metadata (SVM-TIM), contra el modelo propuesto (MDL-TIM) por los autores, se observó que bajo todas las métricas, con excepción de la precisión, el marco de trabajo Multimodal obtuvo mejores resultados, con 0.7505 en sensibilidad, 0.7534 en puntaje F1 y 0.8326 en área bajo la curva, frente a 0.7411, 0.7483 y 0.7411 respectivamente.

\subsection{Undécimo antecedente: «Finding the Keywords Affecting the Success of Crowdfunding Projects» \citep*{pr_chen2019keywords_crowdfunding}}
\citeauthor{pr_chen2019keywords_crowdfunding} realizaron un resumen de la conferencia «2019 IEEE 6th International Conference on Industrial Engineering and Applications (ICIEA)» del 12 al 15 de abril del 2019 en Tokyo, Japón. Este fue titulado \citetitle{pr_chen2019keywords_crowdfunding} la cual traducida al español significa «Encontrando las palabras claves que afectan el éxito de proyectos de financiamiento colectivo».

\subsubsection{Planteamiento del Problema y objetivo}
A la fecha de publicación del acta de congreso del presente antecedente, existían pocas investigaciones basadas en el impacto de la descripción y palabras para predecir el ratio de éxito de financiamiento de un proyecto crowdfunding.

Ante esta situación, los autores tuvieron como objetivo analizar el impacto del contenido textual en un proyecto de Kickstarter a partir del análisis de sus palabras clave que determinen el ratio de éxito de su financiamiento.

\subsubsection{Técnicas empleadas por los autores}
Los autores propusieron 2 modelos de Máquinas de Vectores de Soporte con Eliminación de Características Recursivas (SVM-RFE), usando las herramientas “Weka 3.8” y “libSVM”, los cuales fueron comparados.

\subsubsection{Metodología empleada por los autores}
Se recolectaron las descripciones de proyectos de Kickstarter de la categoría “Juegos”, desde el 1 de julio hasta el 30 de agosto del 2018, de los cuales la distribución quedó en 50 proyectos que alcanzaron el 100\% de su meta y otros 50 proyectos que no alcanzaron por lo menos el 75\%. A continuación, se pre-procesó la data eliminando caracteres especiales, términos distintos al idioma inglés y palabras de parada en inglés para poder armar una matriz de Término-Documento que usará los pesos TF-IDF. Luego, se seleccionaron las características realizando experimentos de validación cruzada antes de la construcción del modelo predictivo. Finalmente, se evaluaron los resultados con la métrica de exactitud.

\subsubsection{Resultados obtenidos}
Se realizó hasta dos experimentos manipulando los conjuntos de datos para el SVM-RFE y el mejor resultado fue una exactitud de 78.90\% para predecir el ratio de éxito de un proyecto crowdfunding a partir de 50 palabras claves importantes que se encuentren dentro de su contenido.

\subsection{Duodécimo antecedente: «The Language that Gets People to Give: Phrases that Predict Success on Kickstarter» \citep*{pr_mitra2014phrases}}
\citeauthor{pr_mitra2014phrases} realizaron un resumen de la conferencia «The 17th ACM conference on Computer supported cooperative work \& social computing (CSCW '14)» del 15 al 19 de febrero del 2014 en Baltimore, Estados Unidos. Este fue titulado \citetitle{pr_mitra2014phrases} la cual traducida al español significa «El lenguaje que hace que la gente dé: Frases que predicen el éxito en Kickstarter».

\subsubsection{Planteamiento del Problema y objetivo}
Existencia de poco conocimiento sobre los factores que impulsan a financiar proyectos a pesar del crecimiento en los últimos años de plataformas de financiamiento colectivo como Kickstarter.

Los autores elaboraron, ante ello, un modelo para predecir el éxito de financiamiento en proyectos crowdfunding a partir de frases textuales.

\subsubsection{Técnicas empleadas por los autores}
Se diseñó un modelo de Regresión logística penalizada para proteger contra la colinealidad y escasez que prevalecen en el conjunto de datos de frases.

\subsubsection{Metodología empleada por los autores}
Se recolectaron 45,815 proyectos de Kickstarter del 2012. A continuación, se realizó limpieza de datos eliminando campañas inconclusas y con esto se procedió a extraer la información textual (descripción y recompensas del proyecto). El siguiente paso fue la generación del cuerpo de más de 9 millones de frases únicas, de las cuales quedaron 20,391 frases que aparecen al menos 50 veces en todos los proyectos. Finalmente, se creó el modelo predictivo para encontrar las frases que determinen que un proyecto será financiado y aquellas que determinen lo contrario.

\subsubsection{Resultados obtenidos}
De las 59 variables de control que no son texto, 29 que tenían «valores aceptables» tenían pesos positivos. 15 variables fueron consideradas para proyectos que serán financiados, mientras que 14 para los que no. Se obtuvo un Top 100 de frases que tenían pesos positivos (proyectos que sí serán financiados), así como también para aquellas con pesos negativos (proyectos que no serán financiados).
El ratio de error del modelo propuesto (variables de control + frases) fue de 2.24.