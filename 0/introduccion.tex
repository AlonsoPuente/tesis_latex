\thispagestyle{plain}
\begin{center}
	%\vspace*{1.5cm}
	{\Large \bfseries  Introducción}
\end{center}
\vspace{0.5cm}

Por muchos años, en especial en las dos últimas décadas, diversos proyectos emprendedores han sido lanzados en distintas plataformas web, buscando un objetivo compartido por todos: ser financiados en un determinado plazo para hacer realidad estas ideas. Entre fracasos y éxitos, han surgido nuevas tendencias, así como nuevos enfoques de estudios de estos casos para encontrar la clave que descifre las variables de éxito.

El presente trabajo de investigación se basó formular un modelo ensamblado robusto que determine el estado final de un proyecto, agregando un nuevo enfoque: basarse solamente en proyectos de tecnología, la segunda categoría con más baja probabilidad de éxito al final de una campaña. En estudios previos, los modelos planteados resultaron bastante aceptables debido a que el resto de categorías de proyectos balancearon la inequidad existente en las dos clases del estado.

El reto principal fue el de construir modelos predictivos que consideren 3 características importantes de un proyecto: la metainformación, descripción de la campaña, y los comentarios acerca de esta hechos por los patrocinadores.

Para ello, se recolectó un total de 27,251 proyectos tecnológicos en Kickstarter entre los periodos 2009-2019. Algunos proyectos provenientes de países fuera del territorio de los Estados Unidos y en distintos idiomas fueron considerados dentro de esta cantidad ya que no afectó al rendimiento general como en casos particulares de algunos estudios previos.

La principal motivación de este trabajo fue la de aportar una herramienta de ayuda para los emprendedores que les permita estimar el estado final del financiamiento de su proyecto de tecnología, es decir, éxito o fracaso, con un nivel confiable de probabilidad de éxito del mismo durante el transcurso de su campaña, permitiendo además servir de soporte en la toma de decisiones de cara a lograr su principal objetivo.