\thispagestyle{plain}
\begin{center}
	%\vspace*{1.5cm}
	{\Large \bfseries  Resumen}
\end{center}
\vspace{0.5cm}

Conocer el destino del financiamiento de proyectos siempre ha sido el principal deseo de todos los emprendedores que los promocionan en Internet, en especial, de la categoría de tecnología por ser los que presentan las ratios más bajas de éxito debido a sus altas metas que buscan alcanzar. El presente trabajo de investigación se basó en construir un modelo predictivo cuyo objetivo es la de estimar el éxito o fracaso de financiamiento de proyectos tecnológicos en la plataforma de crowdfunding Kickstarter durante la duración de su campaña a partir de su metainformación, descripción y contenido social (comentarios). Para ello, se creó un Modelo Apilado Integrado de modelos de Aprendizaje Automático para cada modalidad. Luego de analizar los resultados del modelo final con las mismas métricas mencionadas en los antecedentes, se concluyó que el modelo propuesto funciona mejor de manera en conjunto que por modalidades independientes. Como trabajo a futuro, (COMPLETAR).
%\newline

\textbf{Palabras claves: } metainformación, descripción, imagen del proyecto, Modelo Apilado Integrado, Perceptrón Multicapa (MLP), Red Neuronal Convolucional (CNN).

\vspace{0.5cm}
Knowing funding projects destiny has always been the main desire of all entrepreneurs who promote them on the Internet, especially in the technology category because they have the lowest success rates due to their high goals. The present research work was based on building a predictive model whose objective is to estimate the success or failure of technological projects funding in the Kickstarter crowdfunding platform during the duration of its campaign based on its metadata, description and social media (comments). For this, an Integrated Stacking Model based on Machine Learning models for each modal were created. After analyzing the results with the same metrics mentioned in (TO COMPLETE), it was concluded that (TO COMPLETE). As future work, (TO COMPLETE).
%\newline

\textbf{Palabras claves: } project metadata, project description, project image, Integrated Stacking Model, Multilayer Perceptron (MLP), Convolutional Neural Network (CNN).