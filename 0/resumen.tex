%La línea de abajo es para quitar encabezado
%\thispagestyle{plain}

\chapter*{RESUMEN}
\markboth{RESUMEN}{RESUMEN}
\addcontentsline{toc}{chapter}{RESUMEN}

Conocer el destino del financiamiento de proyectos ha sido el principal deseo de emprendedores que los promocionan en Internet, en especial, de la categoría de tecnología por tener los ratios más bajos de éxito debido a sus altas metas que buscan alcanzar. El presente trabajo de investigación se basó en construir un modelo predictivo cuyo objetivo es identificar el estado de financiamiento de proyectos tecnológicos en la plataforma de crowdfunding Kickstarter durante su campaña basándose en su metainformación, descripción y comentarios de los patrocinadores. Para ello, se creó un modelo ensamblado a partir de otros tres, denominado The Hydra. Al evaluarse bajo las métricas de la literatura, se concluyó que el modelo propuesto presentó mejor rendimiento que su base debido a la mejora de su capacidad colectiva por encima de cada performance individual que lo compone. Como trabajo a futuro, el prototipo funcionará en una plataforma web, más hiperparámetros se ajustarán y se considerarán más variables.
%\newline

\textbf{Palabras claves: } financiamiento colectivo, proyecto, Aprendizaje Profundo Multimodal, Perceptrón Multicapa (MLP), Red Neuronal Convolucional (CNN), Red Neuronal Recurrente (RNN).

\clearpage
%\vspace{0.5cm}
\chapter*{ABSTRACT}
\markboth{ABSTRACT}{ABSTRACT}
Knowing funding projects destiny has been the main desire of entrepreneurs who promote them on the Internet, especially from the technology category for having the lowest success rates due to their high goals. The present research work was based on building a predictive model whose objective is identify the funding state of technology projects from Kickstarter crowdfunding platform during its campaign, basing on its metadata, description and backers comments. To do this, an assembled model was created from three others, called The Hydra. When evaluated under the metrics of the literature, it was concluded that the proposed model presented better performance than its baseline due to the improvement of its collective capacity above each individual performance that composes it. For future work, the prototype will run on a web platform, more hyperparameters will be tuned, and more variables will be considered.
%\newline

\textbf{Keywords: } crowdfunding, project, Multimodal Deep Learning, Multilayer Perceptron (MLP), Convolutional Neural Network (CNN), Recurrent Neural Network (RNN).