\thispagestyle{plain}
\begin{center}
	%\vspace*{1.5cm}
	{\Large \bfseries  Resumen}
\end{center}
\vspace{0.5cm}

Conocer el destino del financiamiento de proyectos siempre ha sido el principal deseo de todos los emprendedores que los promocionan en Internet, en especial, de la categoría de tecnología por ser los que presentan las ratios más bajas de éxito debido a sus altas metas que buscan alcanzar. El presente trabajo de investigación se basó en construir un modelo predictivo cuyo objetivo es la de estimar el éxito o fracaso de financiamiento de proyectos tecnológicos en la plataforma de crowdfunding Kickstarter durante la duración de su campaña a partir de su metainformación, imagen y/o descripción. Para ello, se crearon modelos de Aprendizaje Automático (SVM, MLP y CNN) para cada una de estas partes. Luego de analizar todos los modelos con las mismas métricas mencionadas en el décimo antecedente, se concluyó que solo los de metainformación tuvieron niveles excelentes de acuerdo a sus puntajes AUC (0.8377 para SVM y 0.7043 para MLP), mientras que los modelos de descripciones tuvieron un rendimiento regular (0.6746 para SVM con TF-IDF y 0.6709 para SVM con BoW) y finalmente el modelo de imágenes no logró clasificar las clases correctamente. Como trabajo a futuro, estos últimos modelos de descripción e imagen serán mejorados para construir, junto con los de metainformación, un modelo ensamblado basado en considerables variables del proyecto.
%\newline

\textbf{Palabras claves: } metainformación, descripción, imagen del proyecto, Máquina de Vectores de Soporte (SVM), Perceptrón Multicapa (MLP), Red Neuronal Convolucional (CNN).

\vspace{0.5cm}
Knowing funding projects destiny has always been the main desire of all entrepreneurs who promote them on the Internet, especially in the technology category because they have the lowest success rates due to their high goals. The present research work was based on building a predictive model whose objective is to estimate the success or failure of technological projects funding in the Kickstarter crowdfunding platform during the duration of its campaign based on its metadata, image and/or description. For this, Machine Learning models (SVM, MLP and CNN) were created for each of these parts. After analyzing all the models with the same metrics mentioned in the tenth antecedent, it was concluded that only metadata models had excellent levels according to their AUC scores (0.8377 for SVM and 0.7043 for MLP), while the description models had a regular performance (0.6746 for SVM with TF-IDF and 0.6709 for SVM with BoW) and finally the image model failed to classify the classes correctly. As future work, these latest models of description and image will be improved to build, together with metadata ones, an assembled model base don considerable project variables.
%\newline

\textbf{Palabras claves: } project metadata, project description, project image, Support Vector Machine (SVM), Multilayer Perceptron (MLP), Convolutional Neural Network (CNN).