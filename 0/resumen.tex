%La línea de abajo es para quitar encabezado
%\thispagestyle{plain}

\chapter*{RESUMEN}
\markboth{RESUMEN}{RESUMEN}
\addcontentsline{toc}{chapter}{RESUMEN}

Desde la aparición del crowdfunding, muchas personas, especialmente emprendedoras, han tenido la oportunidad de presentar sus proyectos al público para conseguir su financiamiento. Durante el período 2009-2019, el 37\% de proyectos de Kickstarter, una de las plataformas de financiamiento colectivo más populares, alcanzó ser financiado exitosamente. En trabajos anteriores relacionados con el tema estudiado, se utilizaron distintas metodologías y técnicas de Inteligencia Artificial, considerando todas las categorías existentes en esta plataforma para crear modelos predictivos. Sin embargo, este ratio solo alcanza el 20\% para Tecnología. El objetivo de esta investigación fue predecir el estado de financiamiento de proyectos de tecnología en el sitio web de crowdfunding Kickstarter mediante un modelo de Aprendizaje Profundo Multimodal. Siguiendo la metodología CRISP-DM, se implementó un modelo ensamblado de otros modelos de Aprendizaje Profundo por cada modalidad considerada: un Perceptrón Multicapa para la Metainformación, una Red Neuronal Convolucional para la descripción y un modelo LSTM Bidireccional para los comentarios de los patrocinadores. Se utilizó información de más de 27 mil proyectos de tecnología en Kickstarter entre 2009 y 2019. Al evaluarse por las métricas de clasificación seleccionadas y compararlo contra la línea de base y cada modalidad independiente, la propuesta superó a los otros modelos en cada métrica, alcanzando un valor de 93\% de AUC, la de mejor desempeño. Se logró no solamente desarrollar un modelo predictivo para un problema muy estudiado bajo una nueva perspectiva sino también contribuir con insights y un prototipo analítico para trabajos futuros y apoyo a creadores de proyectos.
%\newline

\textbf{Palabras claves: } financiamiento colectivo, proyecto, Aprendizaje Profundo Multimodal, Perceptrón Multicapa (MLP), Red Neuronal Convolucional (CNN), Red Neuronal Recurrente (RNN).

\clearpage
%\vspace{0.5cm}
\chapter*{ABSTRACT}
\markboth{ABSTRACT}{ABSTRACT}
Since the advent of crowdfunding, many people, especially entrepreneurs, have had the opportunity to present their projects to the public in order to fund them. During the 2009-2019 period, 37\% of Kickstarter projects, one of the most popular crowdfunding platforms, were successfully funded. In previous works related to the subject studied, different Artificial Intelligence methodologies and techniques were used, considering all the existing categories in this platform to develop predictive models of this problem. However, this ratio only reaches 20\% for Technology. The objective of this research was to predict the funding state of technology projects on the Kickstarter crowdfunding website using a Multimodal Deep Learning model. Following the CRISP-DM methodology, an assembled model of other Deep Learning models was implemented for each modal considered: a Multilayer Perceptron for Meta-information, a Convolutional Neural Network for the description and a Bidirectional LSTM model for the comments of the sponsors. Information from more than 27 thousand technology projects on Kickstarter between 2009 and 2019 was used. When evaluated by the selected classification metrics and compared against the baseline and each independent modal, the proposal surpassed the other models in each metric, reaching a value of 93\% of AUC, the one with the best performance. It was possible not only to develop a predictive model for a well-studied problem from a new perspective, but also to contribute insights and an analytical prototype for future work and support to project creators.
%\newline

\textbf{Keywords: } crowdfunding, project, Multimodal Deep Learning, Multilayer Perceptron (MLP), Convolutional Neural Network (CNN), Recurrent Neural Network (RNN).