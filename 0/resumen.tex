\thispagestyle{plain}
\begin{center}
	%\vspace*{1.5cm}
	{\Large \bfseries  Resumen}
\end{center}
\vspace{0.5cm}

Conocer el destino del financiamiento de proyectos siempre ha sido el principal deseo de todos los emprendedores que los promocionan en Internet, en especial, de la categoría de tecnología por ser los que presentan las ratios más bajas de éxito debido a sus altas metas que buscan alcanzar. El presente trabajo de investigación se basó en construir un modelo predictivo cuyo objetivo es la de estimar el éxito o fracaso de financiamiento de proyectos tecnológicos en la plataforma de crowdfunding Kickstarter durante su campaña a partir de su metainformación, descripción y comentarios de los patrocinadores. Para ello, se creó un Modelo Apilado Integrado de modelos de Aprendizaje Profundo para cada modalidad, denominada The Hydra. Luego de analizar los resultados del modelo final con las métricas mencionadas en la literatura, se concluyó que el modelo propuesto presentó mejor rendimiento que la mayoría de los antecedentes debido a la mejora de su capacidad colectiva por encima de cada performance individual que lo compone. Como trabajo a futuro, el sistema piloto que incluye al modelo será insertada en una plataforma web, se afinarán más hiperparámetros y se considerarán más variables potenciales.
%\newline

\textbf{Palabras claves: } metainformación, descripción, comentarios, Aprendizaje Profundo, Modelo Apilado, Perceptrón Multicapa (MLP), Red Neuronal Convolucional (CNN), Red Neuronal Recurrente (RNN).

\vspace{0.5cm}
Knowing funding projects destiny has always been the main desire of all entrepreneurs who promote them on the Internet, especially in the technology category because they have the lowest success rates due to their high goals. The present research work was based on building a predictive model whose objective is to estimate the success or failure of technological projects funding in the Kickstarter crowdfunding platform during its campaign based on its metadata, description and backers comments. For this, an Integrated Stacking Model based on Deep Learning models for each modal created, called The Hydra. After analyzing the final model results with metrics mentioned in the literature, it was concluded that the proposed model presented better performance than most of the antecedents due to the improvement of its collective capacity over each individual performance that composes it. As work in the future, the pilot system that includes the model will be inserted into a web platform, more hyperparameters will be refined and more potential variables will be considered.
%\newline

\textbf{Palabras claves: } project metadata, project description, project comments, Deep Learning, Stacked Model, Multilayer Perceptron (MLP), Convolutional Neural Network (CNN), Recurrent Neural Network (RNN).